%%%%%%%%%%%%%%%%%%%%%%%%%%%%%%%%%%%%%%%%%%%%%%%%%%%%%%%%%%%%%%%%%%%%%%%%%%%
\fancyChapter{Spontaneous Path Tracing in Task-Irrelevant Mazes}
[Spatial Affordances Trigger Dynamic \\ Visual Routines]
\label{chap:JEPG2024}
%%%%%%%%%%%%%%%%%%%%%%%%%%%%%%%%%%%%%%%%%%%%%%%%%%%%%%%%%%%%%%%%%%%%%%%%%%%

% Text copied from https://oce.ovid.com/article/00004785-202409000-00003/HTML

Given a maze (e.g., in a book of puzzles), you might solve it by drawing out paths with your pencil. But even without a pencil, you might naturally find yourself \textit{mentally} tracing along various paths. This “mental path tracing” may intuitively seem to depend on your (overt, conscious, voluntary) goal of wanting to get out of the maze, but might it also occur spontaneously --- as a result of simply \textit{seeing} the maze, via a kind of dynamic visual routine? Here, observers simply had to compare the visual properties of two probes presented in a maze. The maze itself was entirely task irrelevant, but we predicted that simply \textit{seeing} the maze’s visual structure would “afford” incidental mental path tracing (à la Gibson). Across four experiments, observers were slower to compare probes that were further from each other along the paths, even when controlling for lower level properties (such as the probes’ brute linear separation, ignoring the maze “walls”). These results also generalized beyond mazes to other unfamiliar displays with task-irrelevant circular obstacles. This novel combination of two prominent themes from our field --- affordances and visual routines --- suggests that at least some visual routines may not require voluntary goals; instead, they may operate in an automatic (incidental, stimulus-driven) fashion, as a part of visual processing itself.
\clearpage
\begin{kaobox}[frametitle=Significance Statement]
    What goes on in your mind when you solve a maze, for example in a book of puzzles? Normally, this might involve tracing through the paths with a pencil or finger. But not necessarily: You can also solve a maze just by looking. Here we show that this mental path tracing also occurs when you are \textit{not} trying to solve such puzzles (and with unfamiliar sorts of mazes, without entrances or exits): Even when just passively viewing such displays, your mind automatically and spontaneously traces the paths between salient points. We demonstrated this by showing that the time it takes to compare two small probes that appear in a maze (which is itself entirely irrelevant to the task) is a function of how long the path between them is (as opposed to the brute linear distance between them). This reveals how even such simple stimuli may engage surprisingly sophisticated dynamic visual processing.
\end{kaobox}
%%%%%%%%%%%%%%%%%%%%%%%%%%%%%%%%%%%%%%%%%%%%%%%%%%%%%%%%%%%%%%%%%%%%%%%%%%%
\section{Seeing Mazes}\label{sec:seeing_mazes}
%%%%%%%%%%%%%%%%%%%%%%%%%%%%%%%%%%%%%%%%%%%%%%%%%%%%%%%%%%%%%%%%%%%%%%%%%%%
The image shown in \cref{fig:JEPG2024_1} is probably immediately familiar: Most readers, perhaps since childhood, have attempted to solve such mazes --- often by tracing a path from one side to the other with a pencil or crayon. But take a moment now, and reflect on what it is like to simply see such a stimulus. Even without a writing instrument, you may find yourself mentally tracing out various paths through the maze --- and it is certainly possible to solve such a maze only by looking at it. What triggers this sort of mental path tracing? The answer seems obvious: A maze of this kind is a learned cultural artifact made for a specific purpose (as a puzzle to solve), and accordingly the path tracing seems triggered by an overt (conscious, voluntary) goal --- to “get out” of the maze.

\begin{figure}
    \centering
    \includegraphics[width=\textwidth]{figures/JEPG2024/fig1_vectorized2.pdf}
    \caption
    {\textit{An Example Maze}. See text in \cref{sec:seeing_mazes} for details.}
    \label{fig:JEPG2024_1}
\end{figure}

Here, in contrast, we ask whether such mental path tracing may also occur \textit{spontaneously} --- as a result of simply \textit{seeing} the maze --- even in the absence of any explicit goal. Or put in more theoretical terms, we ask whether mazelike stimuli automatically “afford” (à la Gibson) the operation of path tracing as a dynamic visual routine, as a part of visual processing itself. In this way, our exploration of mazes may be of wider interest for the study of perception, insofar as it combines two prominent themes from our field, which have nevertheless not previously been related to each other: \textit{visual routines} and \textit{spatial affordances}.

%%%%%%%%%%%%%%%%%%%%%%%%%%%%%%%%%%%%%%%%%%%%%%%%%%%%%%%%%%%%%%%%%%%%%%%%%%%
\section{Visual Routines}
%%%%%%%%%%%%%%%%%%%%%%%%%%%%%%%%%%%%%%%%%%%%%%%%%%%%%%%%%%%%%%%%%%%%%%%%%%%
Visual processing usually seems both incidental and instantaneous. Spend a moment looking at \cref{fig:JEPG2024_2}, for example, and then keep reading. Now consider the question: What color were the laces? You can probably answer this question immediately, from memory --- indicating that this property was extracted even before you were asked the question, just as a part of natural viewing. (We do not have to \textit{decide} to see colors.) But now consider a different question: Were the green tip and the blue tip part of the same shoelace, or two different shoelaces? You probably do not know the answer yet, which indicates that this property was \textit{not} extracted incidentally during natural viewing. When you look back at \cref{fig:JEPG2024_1}, of course, you can answer this question too --- and you can do so merely by looking (i.e., even without using your finger to follow along a lace). But notice that even here you cannot answer the question immediately: Seeing which tip goes with which seems deliberate, dynamic, and temporally extended (as you mentally “trace” from one tip to another).

The type of visual operation that underlies your ability to answer the which-tip-goes-with-which question when looking at \cref{fig:JEPG2024_2} has been termed a \textit{visual routine} \parencite{ullman_visual_1984,ullman_chapter_1996}. And visual routines contrast with other forms of perception precisely in terms of the two features highlighted in that example: They are often invoked only on demand (rather than always occurring automatically, e.g., “depend[ing] on the goal of the computation”; \cite{ullman_visual_1984}, p.~153), and they are inherently dynamic, such that these operations often take some appreciable time to be executed (as you experience an active “tracing” operation).
\begin{figure}
    \centering
    \includegraphics[width=\textwidth]{figures/JEPG2024/fig2.pdf}
    \caption
    {\textit{Two Shoelaces}. See the online article for the color version of this figure.}
    \label{fig:JEPG2024_2}
\end{figure}

Such visual routines are thought to operate over many types of perceptual queries, especially as they relate to various types of relations --- for example, when determining if one object is \textit{inside} of another contour in the scene, or whether some particular object is to the \textit{right or left} of another particular object. Determining such relationships may seem trivial in scenes with only a few objects, but the dynamic and task-dependent nature of visual routines becomes especially evident when scenes increase in complexity. \cref{fig:JEPG2024_2}, for example, contains only two laces (and thus four tips), but suppose you were looking at a mass of many more laces (perhaps distributed over a wider area): Here you might still be able to answer the which-tip-goes-with-which question “on demand” (and perhaps just as easily), but it would not be as feasible even in principle to precompute such answers prior to the question being posed because there would simply be too many tip pairs to consider. (Similarly, determining if Object 1 is to the left of Object 2 in a two-object scene may seem trivial --- but determining if Object 14 is to the left of Object 87 in a 100-object scene cannot be readily precomputed.) In this way, it would often not be possible for visual routines to operate incidentally (without a specific goal) for reasons of brute computational explosion.

The routine of “curve tracing” has most often been studied in tasks such as the which-tip-goes-with-which question in \cref{fig:JEPG2024_2}, in which observers must determine whether two probes lie upon the same contour (or not). As its name suggests, curve tracing involves dynamic tracking along a contour to determine connectedness. (Sometimes curve tracing is described as an “elemental operation” that can be part of a larger visual routine and other times as a visual routine itself. Here we adopt the latter formulation.) As a result, when participants view displays of nonoverlapping curved contours, response times vary as a function of the length of the contour between the two probes, even while equating their shortest straight-line distance from each other, ignoring the curves (e.g., \cite{jolicoeur_curve_1986}). This result seems robust and generalizable, occurring even in simplified stimuli without excessive curvature \parencite{pringle_mental_1988}, in various other tasks (e.g., where participants must determine whether there are any gaps along the curve between two probes; \cite{jolicoeur_curve_1986}), and even when the curve displays are only flashed briefly \parencite{jolicoeur_visual_1991}.

%%%%%%%%%%%%%%%%%%%%%%%%%%%%%%%%%%%%%%%%%%%%%%%%%%%%%%%%%%%%%%%%%%%%%%%%%%%
\section{Spatial Affordances}
%%%%%%%%%%%%%%%%%%%%%%%%%%%%%%%%%%%%%%%%%%%%%%%%%%%%%%%%%%%%%%%%%%%%%%%%%%%
A separate tradition in vision research has emphasized how seeing an object may often involve spontaneously perceiving how we could interact with it: Could I pick it up? Could I throw it? Could I step over it? This point has most famously been put in terms of \textit{affordances}: “The affordances of the environment are what it offers the animal, what it provides or furnishes, either for good or ill” (\cite{gibson_ecological_1979}, p.~119). In this sense, a small rock (but not a large rock) may afford picking up or throwing, a doorknob (but not a spiderweb) may afford grasping, and a wide aperture (but not a narrower aperture) may afford passing through. Though not as familiar in vision science as are properties such as color or shape, the idea that such affordances (liftability, throwability, graspability, passability) are extracted and used during perception has been tremendously influential (for reviews see \cite{barsingerhorn_possibilities_2012}; \cite{warren_information_2021}), even as the concept has come to be used in many different senses \parencite{chong_evolution_2020}.\footnotemark

\footnotetext{Of course, the wider Gibsonian research program also involved many other components --- including the controversial notion that such properties could be perceived \textbf{directly}, without intervening representations or symbolic computations \parencite{fodor_how_1981, ullman_against_1980}. We address these below in the General Discussion section.}

The notion of “navigational affordances” may be especially relevant in the current contexts of mazes and curve tracing. Recent evidence suggests that when viewing a scene, the occipital place area may encode its traversable paths or routes --- for example, how to get from the current viewpoint to another salient location \parencite{bonner_coding_2017}. And critically, at least three forms of recent evidence suggest that such navigational affordances are extracted spontaneously during scene perception. First, such routes are encoded even when an observer’s task has nothing to do with navigation (e.g., when they are simply detecting colors of two probes; \cite{bonner_coding_2017}). Second, event-related potential evidence suggests that such representations (e.g., of the number of doors and thus possible paths) are formed within \qty{200}{\milli\second} of viewing a scene \parencite{harel_early_2022}. And third, recent behavioral evidence demonstrates that a particular type of navigational affordance --- a path to a salient \textit{exit} from an enclosed space --- is extracted even when task irrelevant: When observers must simply detect any change (of any type) to an image, obstacles that move to block (or unblock) the path to an exit are more readily noticed than those that do not \parencite{belledonne_automatic_2021, belledonne_navigational_2022}.

%%%%%%%%%%%%%%%%%%%%%%%%%%%%%%%%%%%%%%%%%%%%%%%%%%%%%%%%%%%%%%%%%%%%%%%%%%%
\section[The Present Study]{The Present Study: Spontaneous Visual Routines Triggered by Affordances?}
%%%%%%%%%%%%%%%%%%%%%%%%%%%%%%%%%%%%%%%%%%%%%%%%%%%%%%%%%%%%%%%%%%%%%%%%%%%
The role of conscious, voluntary goals has often been emphasized in theoretical work on the nature of visual routines (\cite{ullman_visual_1984}; \cite{ullman_chapter_1996}). And past empirical work on curve tracing has to our knowledge almost always inherently involved an explicit goal --- for example, when participants are explicitly asked to determine whether two probes are on the same contour (in a display with multiple contours; \cite{jolicoeur_curve_1986}; cf. \cite{pringle_mental_1988}).\footnotemark

\footnotetext{\label{fn:JEPG2024_2}To our knowledge, the only partial exception to this involves Experiments 3 and 4 from \cite{pringle_mental_1988}. In these experiments, subjects indicated not whether two probes were on the same contour but rather whether two probes were on the same side of a gap between two contours. On one hand, this avoids explicit reference to the“same contour.”But on the other hand, it still seems to involve a more implicit task demand to note the positions of the probes (to determine their spatial relationships to the gap, which was in turn defined by the curves themselves). In contrast, to foreshadow, the current experiments involve probes whose spatial locations are entirely task irrelevant.} Here, in contrast, we ask whether this sort of dynamic visual routine may instead operate in a more automatic and spontaneous fashion, as a part of seeing itself, even without any explicit goal (perhaps more akin to the perception of color). In other words, we ask whether certain stimuli effectively \textit{afford} the spontaneous operation of certain visual routines. In particular, we ask in empirical terms whether the navigational affordances of 2D mazelike stimuli (with distinct paths and barriers) are themselves sufficient to trigger mental path tracing, even when they are entirely task irrelevant.

\begin{figure}
    \centering
    \includegraphics[width=\textwidth]{figures/JEPG2024/fig3v3.pdf}
    \caption
    [\textit{A Zoomed-In Example of (a) a Pathwise Near Probe Trial and (b) Its Corresponding Pathwise Far Probe Trial}]
    {\textit{A Zoomed-In Example of (a) a Pathwise Near Probe Trial and (b) Its Corresponding Pathwise Far Probe Trial}. In each case, the probes have an equal straight-line distance from each other but (a) has a pathwise distance of seven steps, while (b) has a path wise distance of 10 steps. In (a) the two probes are identical (with their white cutout “notches” both at the bottom), but in (b) the two probes are different (with their notches in different positions). The examples provided here both have three turns in the path when tracing from one probe to the other. A fully zoomed-out version of this maze is provided in the Supplemental Material file. See the online article for the color version of this figure.}
    \label{fig:JEPG2024_3}
\end{figure}
In Experiment 1, observers view mazelike stimuli (though without the entrances and exits of typical mazes) upon which two small probes (each a small blue square with a white cutout) appear (see \cref{fig:JEPG2024_3}), and they must simply indicate whether the two probes are identical or not. The mazes are thus entirely task irrelevant, but we show that they nevertheless matter: Observers’ response times depend not just on the linear interprobe distance but also on the “pathwise” distance between them, along the paths of the maze --- an effect we interpret in terms of spontaneous mental path tracing. Experiments 2a and 2b each replicate this effect while varying various aspects of the stimuli and the probe timing. Finally, Experiment 3 generalizes these effects to other forms of spatial paths and obstacles --- illustrating that spontaneous mental path tracing operates as a part of visual processing itself and is not dependent on the familiarity of cultural artifacts such as mazes.


%%%%%%%%%%%%%%%%%%%%%%%%%%%%%%%%%%%%%%%%%%%%%%%%%%%%%%%%%%%%%%%%%%%%%%%%%%%
\subsection{Transparency and Openness}
%%%%%%%%%%%%%%%%%%%%%%%%%%%%%%%%%%%%%%%%%%%%%%%%%%%%%%%%%%%%%%%%%%%%%%%%%%%
The raw data for all studies are included in the Supplemental Material file. The preregistered methods and analyses for each experiment are available at \prettyurl{https://aspredicted.org/L3Q_KFC} (for Experiment 1), \prettyurl{https://aspredicted.org/DSM_NK9} (for Experiment 2a), \prettyurl{https://aspredicted.org/J3F_JX7} (for Experiment 2b), and \prettyurl{https://aspredicted.org/Y55_7YG} (for Experiment 3).

%%%%%%%%%%%%%%%%%%%%%%%%%%%%%%%%%%%%%%%%%%%%%%%%%%%%%%%%%%%%%%%%%%%%%%%%%%%
\section{Experiment 1: Task-irrelevant Mazes}
%%%%%%%%%%%%%%%%%%%%%%%%%%%%%%%%%%%%%%%%%%%%%%%%%%%%%%%%%%%%%%%%%%%%%%%%%%%
Observers were repeatedly shown displays with two probes --- each of which was a blue square with a cutout that could appear along one edge --- and they simply had to indicate whether the probes were identical or not (i.e., whether the cutout was in the same position). These probes were presented in the context of mazelike stimuli (as in \cref{fig:JEPG2024_3}) that were entirely task irrelevant. We asked whether response times would nevertheless vary as a function of the pathwise distance through the mazes between the probes, even when contrasting cases have the same straight-line distance.

%%%%%%%%%%%%%%%%%%%%%%%%%%%%%%%%%%%%%%%%%%%%%%%%%%%%%%%%%%%%%%%%%%%%%%%%%%%
\subsection{Method}
%%%%%%%%%%%%%%%%%%%%%%%%%%%%%%%%%%%%%%%%%%%%%%%%%%%%%%%%%%%%%%%%%%%%%%%%%%%

%%%%%%%%%%%%%%%%%%%%%%%%%%%%%%%%%%%%%%%%%%%%%%%%%%%%%%%%%%%%%%%%%%%%%%%%%%%
\subsubsection{Participants}
%%%%%%%%%%%%%%%%%%%%%%%%%%%%%%%%%%%%%%%%%%%%%%%%%%%%%%%%%%%%%%%%%%%%%%%%%%%
Fifty observers (35 males, 15 females, $M\textsubscript{age} = 23.34$~years) participated using the Prolific online platform \parencite{palan_prolificacsubject_2018} for monetary compensation, with this preregistered sample size determined before data collection began. (For preregistration details, see the Transparency and Openness section. We had no a priori basis for expecting any particular effect size largely because no previous experiments have tested such questions. So we chose our initial preregistered sample size simply to match those from related experiments run with similar online samples from our lab.) Gender information in all experiments was collected via a multiple-choice question, with the options of “male,” “female,” or “other.” Observers were excluded (with replacement) according to two preregistered criteria. First, in a postexperimental debriefing phase, observers self-reported how well they paid attention (on a continuous scale, between 1 = \textit{very distracted} and 100 = \textit{very focused}), and we excluded observers who self-reported an attention level below 75 ($n = 13$). Second, we also excluded observers (who were not already excluded via criterion 1) whose overall accuracy was less than 80\%~, or whose mean response time was at least \qty{2}{SD}s from the mean response time of all observers ($n = 11$).

%%%%%%%%%%%%%%%%%%%%%%%%%%%%%%%%%%%%%%%%%%%%%%%%%%%%%%%%%%%%%%%%%%%%%%%%%%%
\subsubsection{Apparatus}
%%%%%%%%%%%%%%%%%%%%%%%%%%%%%%%%%%%%%%%%%%%%%%%%%%%%%%%%%%%%%%%%%%%%%%%%%%%

After agreeing to participate, observers were redirected to a website where stimulus presentation and data collection were controlled via custom software written using a combination of HTML, CSS, JavaScript, PHP, and the jsPsych libraries \parencite{de_leeuw_jspsych_2023}. Observers completed the experiment on either a laptop or desktop computer. (Since the experiment was rendered on observers’ own web browsers, viewing distance, screen size, and display resolutions could vary dramatically, so we report stimulus dimensions below using pixel [\unit{\pixel}] values.)

%%%%%%%%%%%%%%%%%%%%%%%%%%%%%%%%%%%%%%%%%%%%%%%%%%%%%%%%%%%%%%%%%%%%%%%%%%%
\subsubsection{Stimuli}
%%%%%%%%%%%%%%%%%%%%%%%%%%%%%%%%%%%%%%%%%%%%%%%%%%%%%%%%%%%%%%%%%%%%%%%%%%%

All text, across the instructions and prompts, was presented in a modified version of jsPsych’s default CSS style: black text on a white background drawn in the “Open Sans” font, presented at a font size that is scaled to 2.4\% of the participant’s viewport height.
Maze stimuli were constructed via custom Python code. Mazes were constructed from a $31\times 31$ square grid, each grid square being $26\times\qty{26}{\pixel}$, and with some grid squares colored black to form “walls”. The entire perimeter of the grid was also black. As in the zoomed-in example from \cref{fig:JEPG2024_3}, this yielded a mazelike stimulus but without an “entrance” or “exit.”
Each probe consisted of a blue (hex No. 0000ff~\legendbox{0000ff}) square ($15\times\qty{15}{\pixel}$) with a small rectangular segment ($7\times\qty{3}{\pixel}$) cutout of one of its sides (top, bottom, left, or right). This cutout was not centered on the edge of the square but was rather placed with its center \qty{5.5}{\pixel} away from one corner (and thus \qty{9.5}{\pixel} away from the other corner).

%%%%%%%%%%%%%%%%%%%%%%%%%%%%%%%%%%%%%%%%%%%%%%%%%%%%%%%%%%%%%%%%%%%%%%%%%%%
\subsubsection{Procedure and Design}
%%%%%%%%%%%%%%%%%%%%%%%%%%%%%%%%%%%%%%%%%%%%%%%%%%%%%%%%%%%%%%%%%%%%%%%%%%%

Each trial began with the mazelike stimulus (henceforth “the maze”) appearing at the center of the display. Two probes then appeared sequentially within the maze, the first after \qty{800}{ms} and the second after an additional delay randomly chosen between 650 and \qty{700}{ms}. Both probes appeared along a (white) maze path (as in \cref{fig:JEPG2024_3}), placed as described below. Each probe had a cutout along the same edge of its square. On half of the trials, these cutouts were in the same position along the edge (as in \cref{fig:JEPG2024_3}a), while on other trials, the cutouts were placed in different positions (as in \cref{fig:JEPG2024_3}b). (For example, for probes with cutouts on the right side of the square, the cutout could be either near the top edge of the right side or near the bottom edge of the right side.) On each trial, the full stimulus (maze probes together) was randomly rotated by \qty{0}{\degree}, \qty{90}{\degree}, \qty{180}{\degree}, or \qty{270}{\degree} and was randomly flipped across its horizontal axis, its vertical axis, both axes, or neither axis. Both probes and the maze remained on screen until the observer gave a valid keypress to indicate whether the two probes were identical or not (in terms of the placement of the cutout).

The placements of the probes varied across trials in a systematic way. Throughout the experiment, the trials always came in matched pairs in which the two probes had the same linear distance from each other, placed within the same maze. However, the pathwise distance between the probes was different for each trial in a pair, with one (having “Near” probes, as in \cref{fig:JEPG2024_3}a) being shorter than the other (having “Far” probes, as in \cref{fig:JEPG2024_3}b). Each observer completed 80 trials, presented in a different random order: 4 Mazes $\times$ 2 probe pairs (Far probes vs. Near probes) $\times$ 2 probe matching possibilities (identical vs. different) $\times$ 5 repetitions. (Across the four base mazes, both the straight-line distance and the pathwise distance varied.) Per the preregistered criteria, we excluded individual trials whose response times were more than $2~SDs$ from the mean response time of all observers (on average 1.58 out of 80 trials/observer).

%%%%%%%%%%%%%%%%%%%%%%%%%%%%%%%%%%%%%%%%%%%%%%%%%%%%%%%%%%%%%%%%%%%%%%%%%%%
\subsection{Results and Discussion}
%%%%%%%%%%%%%%%%%%%%%%%%%%%%%%%%%%%%%%%%%%%%%%%%%%%%%%%%%%%%%%%%%%%%%%%%%%%

As depicted in \cref{fig:JEPG2024_4}a, response times were significantly greater on Far probe trials than on Near probe trials, as confirmed with a two-tailed paired t test, \qty{1284.87}{\milli\second} versus \qty{1232.97}{\milli\second}, $t(49) = 4.06$, $p < .001$, $d = .58$. Because the only difference between these two trial types involved the pathwise distance between the probes as they were situated in the mazes, this suggests that observers were mentally tracing through the paths, despite their task irrelevance.
\begin{figure}
    \centering
    \includegraphics[width=\textwidth]{figures/JEPG2024/fig4.pdf}
    \caption
    [\textit{Average Response Times in (a) Experiment 1, (b) Experiment 2a, and (c) Experiment 2b}]
    {\textit{Average Response Times in (a) Experiment 1, (b) Experiment 2a, and (c) Experiment 2b}. Error bars reflect 95\% confidence intervals after subtracting shared variance. See the online article for the color version of $^{*}p < .05$, $^{**}p < .01$, $^{***}p < .001$ (significant differences between conditions).}
    \label{fig:JEPG2024_4}
\end{figure}


%%%%%%%%%%%%%%%%%%%%%%%%%%%%%%%%%%%%%%%%%%%%%%%%%%%%%%%%%%%%%%%%%%%%%%%%%%%
\section{Experiment 2a: Turn-Equated Paths}
%%%%%%%%%%%%%%%%%%%%%%%%%%%%%%%%%%%%%%%%%%%%%%%%%%%%%%%%%%%%%%%%%%%%%%%%%%%
The pattern of results observed in Experiment 1 cannot be explained without reference to the maze paths because the straight-line distance between the probes was always equated across trial pairs. But there are still other ways in which the (necessarily different) maze paths could have influenced response times on Far probe versus Near probe trials. Rather than reflecting influences of pathwise distances (as we hypothesize), for example, the difference could be due only to some alternative factor --- such as a differential number of turns along those paths. We assess this here in a conceptual replication by maintaining longer versus shorter pathwise distances while equating the number of turns (as in \cref{fig:JEPG2024_3}, where both the Far probe and Near probe paths have three turns).

%%%%%%%%%%%%%%%%%%%%%%%%%%%%%%%%%%%%%%%%%%%%%%%%%%%%%%%%%%%%%%%%%%%%%%%%%%%
\subsection{Method}
%%%%%%%%%%%%%%%%%%%%%%%%%%%%%%%%%%%%%%%%%%%%%%%%%%%%%%%%%%%%%%%%%%%%%%%%%%%

This experiment was identical to Experiment 1, except as noted. A new set of 100 observers (50 males, 50 females, $M\textsubscript{age} =  26.08$) were recruited, with this preregistered sample size chosen before data collection began. (This sample size was arbitrarily doubled relative to Experiment 1 on the basis of the experimental design: While the results of Experiment 1 were robust, the current experiment also controlled for additional factors that were not measured or manipulated in Experiment 1, as described below.) Observers were excluded (with replacement) according to the same two preregistered criteria: self-reported attention level ($n = 26$) and outlying overall performance ($n = 13$). Stimuli were identical to those used in Experiment 1, except that mazes spanned a $31 \times 15$ grid, and matching Far and Near trials were generated with the new restriction that the interprobe paths had the same number of turns. (Mazes and probes were still randomly reflected as in Experiment 1, but they were not randomly rotated due to the new rectangular dimensions of the grid.) Per the preregistered criteria, we excluded individual trials whose response times were more than \qty{2}{SD}s from the mean response time of that observer (on average 3.56 out of 80 trials/observer).

%%%%%%%%%%%%%%%%%%%%%%%%%%%%%%%%%%%%%%%%%%%%%%%%%%%%%%%%%%%%%%%%%%%%%%%%%%%
\subsection{Results and Discussion}
%%%%%%%%%%%%%%%%%%%%%%%%%%%%%%%%%%%%%%%%%%%%%%%%%%%%%%%%%%%%%%%%%%%%%%%%%%%

As depicted in \cref{fig:JEPG2024_4}b, response times were again significantly greater on Far probe trials than on Near probe trials, \qty{1282.50}{\milli\second} versus \qty{1259.07}{\milli\second}, $t(99) = 2.55$, $p = .012$, $d = .25$. These results effectively replicate those from Experiment 1 while confirming that the observed difference must reflect spontaneous path tracing through the task-irrelevant mazes. (We note that the effect size in this experiment was smaller than that from Experiment 1, suggesting that other factors such as the number of turns may also have influenced the earlier results --- but of course the key point is that such factors cannot explain, or even contribute to, the current results because they were carefully equated.)

%%%%%%%%%%%%%%%%%%%%%%%%%%%%%%%%%%%%%%%%%%%%%%%%%%%%%%%%%%%%%%%%%%%%%%%%%%%
\section{Experiment 2b: Simultaneous Probe Presentation}
%%%%%%%%%%%%%%%%%%%%%%%%%%%%%%%%%%%%%%%%%%%%%%%%%%%%%%%%%%%%%%%%%%%%%%%%%%%
In Experiments 1 and 2a, the two probes on each trial appeared sequentially. Could this have somehow encouraged mental path tracing (as attention was captured by the second probe, having already been focused on the first probe)? We explore this here in another conceptual replication, in which both probes appeared simultaneously.

%%%%%%%%%%%%%%%%%%%%%%%%%%%%%%%%%%%%%%%%%%%%%%%%%%%%%%%%%%%%%%%%%%%%%%%%%%%
\subsection{Method}
%%%%%%%%%%%%%%%%%%%%%%%%%%%%%%%%%%%%%%%%%%%%%%%%%%%%%%%%%%%%%%%%%%%%%%%%%%%
This experiment was identical to Experiment 2a, except as noted. A new set of 100 observers (61 males, 39 females, $M\textsubscript{age} = 28.44$) were recruited, with this preregistered sample size chosen before data collection began. Observers were excluded (with replacement) according to the same two preregistered criteria ($n = 14$ and $n = 17$, respectively). Stimuli were identical to those used in Experiment 2a, except that both probes appeared simultaneously. Per the preregistered criteria, we again excluded individual trials whose response times were more than $2~SDs$ from the mean response time of that observer (on average 3.33 out of 80 trials/observer).

%%%%%%%%%%%%%%%%%%%%%%%%%%%%%%%%%%%%%%%%%%%%%%%%%%%%%%%%%%%%%%%%%%%%%%%%%%% 
\subsection{Results and Discussion}
%%%%%%%%%%%%%%%%%%%%%%%%%%%%%%%%%%%%%%%%%%%%%%%%%%%%%%%%%%%%%%%%%%%%%%%%%%%
As depicted in \cref{fig:JEPG2024_4}c, response times were again significantly greater on Far probe trials than on Near probe trials, \qty{1304.04}{\milli\second} versus \qty{1282.68}{\milli\second}, $t(99) = 3.14$, $p = .002$, $d = .31$. These results effectively replicate those from Experiment 1 while demonstrating that the spontaneous mental path tracing was not somehow triggered by (and does not require) sequential probe presentation.

%%%%%%%%%%%%%%%%%%%%%%%%%%%%%%%%%%%%%%%%%%%%%%%%%%%%%%%%%%%%%%%%%%%%%%%%%%%
\section{Experiment 3: Beyond Mazes}
%%%%%%%%%%%%%%%%%%%%%%%%%%%%%%%%%%%%%%%%%%%%%%%%%%%%%%%%%%%%%%%%%%%%%%%%%%%
The mental path tracing observed in the previous three experiments was spontaneous insofar as the mazes themselves were completely task irrelevant. Nevertheless, the mazelike nature of the stimuli could have been essential in two ways. First, the fact that the visible “walls” of the maze were all vertical and horizontal could have somehow encouraged people to shift their attention between the probes along horizontal and vertical paths. Second, despite their lack of entrances and exits, those stimuli were still clearly recognizable as mazes. As such, it remains possible that the mental path tracing reflects a sort of ingrained habit, due to the cultural familiarity of the stimulus, despite its irrelevance to the task.
To explore these possibilities, we replicated our study using a completely different --- and completely unfamiliar --- sort of visual scene. Observers completed the same task, but now the probes were presented amid task-irrelevant displays of discs with various sizes and colors, as depicted in \cref{fig:JEPG2024_5}. As before, the straight-line distance between probes was equated across pairs of trials, which nevertheless had Near or Far probe paths that avoided the discs. (And this new type of display also allowed us to control for several other factors, such as the number of disc boundaries that the straight-line path would intersect, and the total amount of overlap between the discs and the straight-line paths.)


%%%%%%%%%%%%%%%%%%%%%%%%%%%%%%%%%%%%%%%%%%%%%%%%%%%%%%%%%%%%%%%%%%%%%%%%%%%
\subsection{Method}
%%%%%%%%%%%%%%%%%%%%%%%%%%%%%%%%%%%%%%%%%%%%%%%%%%%%%%%%%%%%%%%%%%%%%%%%%%%
This experiment was identical to Experiment 2a except as noted.

%%%%%%%%%%%%%%%%%%%%%%%%%%%%%%%%%%%%%%%%%%%%%%%%%%%%%%%%%%%%%%%%%%%%%%%%%%%
\subsubsection{Participants}
%%%%%%%%%%%%%%%%%%%%%%%%%%%%%%%%%%%%%%%%%%%%%%%%%%%%%%%%%%%%%%%%%%%%%%%%%%%
A new set of 100 observers (60 males, 40 females, $M_\text{age} = 27.07$) participated, with this preregistered sample size determined before data collection began. Observers were excluded (with replacement) according to the same two preregistered criteria ( $n = 5$ and $n = 9$, respectively).

%%%%%%%%%%%%%%%%%%%%%%%%%%%%%%%%%%%%%%%%%%%%%%%%%%%%%%%%%%%%%%%%%%%%%%%%%%%
\subsubsection{Stimuli}
%%%%%%%%%%%%%%%%%%%%%%%%%%%%%%%%%%%%%%%%%%%%%%%%%%%%%%%%%%%%%%%%%%%%%%%%%%%
Displays each contained four discs, each with a \qty{1}{\pixel} outline: a green (No. 438c59~\legendbox{438c59}) disc with a dark green (No. 1f4913~\legendbox{1f4913}) outline (radius \qty{71}{\pixel}), a magenta (No. 8c4367~\legendbox{8c4367}) disc with a dark magenta (No. 441048~\legendbox{441048}) outline (radius \qty{64.5}{\pixel}), a purple (No. 5f438c~\legendbox{5f438c}) disc with a dark purple (No. 141b49~\legendbox{141b49}) outline (radius \qty{55}{\pixel}), and a yellow (No. 8c8543~\legendbox{8c8543}) disc with a dark yellow (No. 492814~\legendbox{492814}) outline (radius \qty{51.5}{\pixel}). These discs were placed in a central $600\times \qty{600}{\pixel}$ region as described below, with a minimum spacing (between their nearest edges) of \qty{13}{\pixel}. Probes always appeared diagonally from each other, along roughly opposite diagonals, with the first probe appearing (again after \qty{800}{\milli\second}) near the top and the second probe appearing (always after an additional \qty{400}{\milli\second}) near the bottom (so that, e.g., if the first probe was positioned toward the upper right corner of the central region, the second probe was positioned toward the lower left corner).

%%%%%%%%%%%%%%%%%%%%%%%%%%%%%%%%%%%%%%%%%%%%%%%%%%%%%%%%%%%%%%%%%%%%%%%%%%%
\subsubsection{Procedure and Design}
%%%%%%%%%%%%%%%%%%%%%%%%%%%%%%%%%%%%%%%%%%%%%%%%%%%%%%%%%%%%%%%%%%%%%%%%%%%
As in the previous experiments, the placements of the probes relative to the discs varied across trials in a systematic way --- again using matched pairs of trials in which the two probes had the same linear distance from each other. Here, in addition, the two probes in a matched pair of trials appeared in identical positions on the display, with only the placements of the discs being varied --- such that curving paths between the same two probe positions (i.e., to avoid touching the discs) could be either Near (as in \cref{fig:JEPG2024_5}a) or Far (as in \cref{fig:JEPG2024_5}b). (In fact, these different disc placements were always created simply by swapping the positions of two of the discs. In \cref{fig:JEPG2024_5}, e.g., you can see that the positions of the purple and magenta discs have been swapped, while the positions of the green and yellow discs are identical. And these particular placements were chosen so that the intersections between linear interprobe paths and discs were always equated across the two trials in a pair, in terms of both the number of intersecting discs and the total extent of the intersection.)

Each observer completed 32 trials, presented in a different random order: 2 probe pairs (Far probes vs. Near probes) $\times$ 2 possible reflections (horizontal reflection vs. no reflection) $\times$ 2 probe matching possibilities (identical vs. different) $\times$ 4 repetitions. Within this design, each observer was randomly assigned to one of 4 baseline displays, each with a different arrangement of discs. The first three trials were practice trials, the results of which were not recorded. Per the preregistered criteria, we again excluded individual trials whose response times were more than \qty{2}{SD}s from the mean response time of that observer (on average 1.31 out of 32 trials/observer).
\begin{figure}
    \centering
    \includegraphics[width=\textwidth]{figures/JEPG2024/fig5.pdf}
    \caption
    [\textit{Sample Stimuli and Results From Experiment 3}]
    {\textit{Sample Stimuli and Results From Experiment 3}. (a) a pathwise Near probe trial and (b) its corresponding pathwise Far probe trial. In each case, the probes have an equal straight-line distance from each other, but (a) has a shorter pathwise distance than (b). (c) Average response times. Error bars reflect \qty{95}{\percent} confidence intervals after subtracting shared variance. See the online article for the color version of this figure. $^{*}p < .05$ (significant differences between conditions).}
    \label{fig:JEPG2024_5}
\end{figure}

%%%%%%%%%%%%%%%%%%%%%%%%%%%%%%%%%%%%%%%%%%%%%%%%%%%%%%%%%%%%%%%%%%%%%%%%%%%
\subsection{Results and Discussion}
%%%%%%%%%%%%%%%%%%%%%%%%%%%%%%%%%%%%%%%%%%%%%%%%%%%%%%%%%%%%%%%%%%%%%%%%%%%

As depicted in \cref{fig:JEPG2024_5}c, response times were again significantly greater on Far probe trials than on Near probe trials, \qty{1016.21}{\milli\second} versus \qty{996.74}{\milli\second}, $t (99) = 2.18$, $p = .032$, $d = .22$. These results effectively replicate all of the earlier experiments, showing that such effects generalize to very different displays, which (a) do not have salient rectilinear paths and (b) do not constitute familiar cultural artifacts (such as mazes).


%%%%%%%%%%%%%%%%%%%%%%%%%%%%%%%%%%%%%%%%%%%%%%%%%%%%%%%%%%%%%%%%%%%%%%%%%%%
\section{General Discussion}
%%%%%%%%%%%%%%%%%%%%%%%%%%%%%%%%%%%%%%%%%%%%%%%%%%%%%%%%%%%%%%%%%%%%%%%%%%%
The central empirical result of the four experiments reported here is straightforward: People seemed to mentally trace paths between salient landmarks in mazelike stimuli, even when there was no need to do so --- since the mazes themselves were entirely task irrelevant. In particular, observers were faster to compare two probes (presented either sequentially or simultaneously) when the \textit{pathwise} distance (i.e., through the maze) between them was shorter. These results were first obtained (in Experiments 1, 2a, and 2b) using stimuli that were evocative of common cultural artifacts (similar to mazes in puzzle books, as in \cref{fig:JEPG2024_3}), but they also generalized to entirely unfamiliar stimuli (with circular obstacles between probes, as in \cref{fig:JEPG2024_5}). And critically, these results had to reflect the paths between the objects (avoiding crossing over the walls of the maze, or the circular obstacles) because other factors were held constant --- for example, the straight-line distances between the probes (in all experiments), the number of turns required along the maze paths (in Experiments 2a and 2b), and the exact degree to which straight-line paths intersected with circular obstacles (in Experiment 3). (These other factors may still play independent roles in response times, but they cannot explain the key path-length effects.)

In theoretical terms, we have described these results via a combination of two foundational themes from vision research: \textit{visual routines} (à la Ullman) and \textit{affordances} (à la Gibson). In this context, the current results suggest that certain types of spatial affordances automatically trigger the operation of certain visual routines, in a spontaneous stimulus-driven manner. (This is consistent with an earlier suggestion from \cite{pringle_mental_1988}, that curve tracing “might well be automatic” [p.~726]; cf. \cref{fn:JEPG2024_2}.) This conclusion extends our understanding of both visual routines and affordances in certain critical ways --- in each case suggesting that certain key features of these frameworks can be divorced from others.

%%%%%%%%%%%%%%%%%%%%%%%%%%%%%%%%%%%%%%%%%%%%%%%%%%%%%%%%%%%%%%%%%%%%%%%%%%%
\subsection[Expanding Visual Routines]{Expanding Visual Routines (Dynamic Visual Operations, but Without Voluntary Goals?)}
%%%%%%%%%%%%%%%%%%%%%%%%%%%%%%%%%%%%%%%%%%%%%%%%%%%%%%%%%%%%%%%%%%%%%%%%%%%
Perhaps the two key properties of visual routines, as reviewed above, are that they are (a) inherently dynamic and thus temporally extended while also (b) operating only “on demand” (as when determining which tip goes with which in \cref{fig:JEPG2024_2}; \cite{ullman_visual_1984}; \cite{ullman_chapter_1996}) --- as opposed to visual processes that occur seemingly instantaneously without any specific goal (as when determining what color the laces were in \cref{fig:JEPG2024_2}). These features are both salient properties of traditional visual routines, even beyond path tracing. When looking at a complex mass of twisting contours, for example, you can tell whether a given dot is inside a closed contour or not (as in Fig.~2c of \cite{ullman_visual_1984}), and you can tell just by looking, but (a) it takes an appreciable amount of time to determine this (via the operation of “coloring”), and (b) this only occurs when you have the explicit intention to determine this inside/outside relation --- which may not occur at all while you casually view such a stimulus.

Here we suggest that these two properties can be divorced from each other. In particular, the current experiments with mazes have plenty of synergy with the first property, but not with the second: These results suggest that the visual routine of path tracing may \textit{not} require explicit, voluntary goals; instead, it may operate (still in an inherently dynamic way) in a \textit{spontaneous} manner, as a part of visual processing --- with the “demand” being triggered by the spatial affordances of the stimuli themselves. (Of course, in this context, spontaneity does not imply irresistibility: Observers might still be able to compare two probes without tracing along the paths if they intentionally tried to do so; what the current results show, however, is that such path tracing occurs by default, without any explicit intention.)

Putting this same point in more operationalized terms, note that previous experiments on the operation of curve tracing could not speak to this possibility because they always involved an explicit, voluntary goal that \textit{required} the curves to be traced (e.g., to determine whether two probes were on the same curve; \cite{jolicoeur_curve_1986,jolicoeur_visual_1991, mccormick_capturing_1992}; cf.~\cite{pringle_mental_1988} and see \cref{fn:JEPG2024_2}). In contrast, in the present experiments, not only did observers \textit{not} have the explicit goal of path tracing (since the mazes were task irrelevant), but the operation of this visual routine may have actually harmed performance: Observers could have simply “traced” in a straight line from one probe to the other --- ignoring the irrelevant walls or discs --- yet instead they effectively took unnecessary mental detours around these obstacles.

%%%%%%%%%%%%%%%%%%%%%%%%%%%%%%%%%%%%%%%%%%%%%%%%%%%%%%%%%%%%%%%%%%%%%%%%%%%
\subsection[Expanding Affordances]{Expanding Affordances (Functional Stimulus-Driven Percepts, but Without Direct Perception?)}
%%%%%%%%%%%%%%%%%%%%%%%%%%%%%%%%%%%%%%%%%%%%%%%%%%%%%%%%%%%%%%%%%%%%%%%%%%%
The notion of affordances is famously plastic \parencite{chong_evolution_2020}, but perhaps their two key properties in most interpretations (as inspired by the work of Gibson; e.g., \cite{gibson_ecological_1979}) are that they are (a) stimulus-driven percepts that go beyond low-level features while also (b) being perceived “directly.” (In some conceptions, they are also intrinsically linked to potential \textit{actions}.) The maze stimuli in the present experiments have both low-level features (various contours of different orientations) and seemingly higher level functional features (such as the “traversability” between two probes in the maze). For Gibson it is the latter sort of property that is primary because it is what functionally matters to an organism: "Psychologists assume that objects are composed of their qualities. But I now suggest that what we perceive when we look at objects are their affordances, not their qualities. We can discriminate the dimensions of difference if required to do so in an experiment, but what the object affords us is what we normally pay attention to" (\cite{gibson_ecological_1979}, p.~126).

At the same time, however, the perception of such affordances is thought to be \textit{direct} --- a function of brute “resonance” with the environment, in a type of “information pickup,” without intermediate inferential steps. This notion of “directness” is notoriously controversial and difficult to unpack --- as it has been argued that there is simply no feasible story for how this “pickup” could occur without mediating representations and intermediate computational steps \parencite{fodor_how_1981, ullman_against_1980}. But at least one animating aspect of direct perception is that it is thought to be \textit{immediate}: “there can be direct or immediate awareness of objects and events when the perceptual system resonates so as to pick up information” (\cite{gibson_ecological_1979}, p.~168).

Here we suggest that these two properties can be divorced from each other. In particular, the current experiments with mazes have plenty of synergy with the first property, but not with the second. These results suggest that simply viewing the mazelike stimuli themselves (without any particular goal) already involves the extraction not only of various lower level features (e.g., the colors of the contours) but also of higher level functional properties, in the form of navigational affordances. This is just the same sense in which, for Gibson, viewing a stone involves the extraction not only of its various lower level features (e.g., its color) but also of affordances such as its “throwability” and “graspability”. However, the perception of such properties in the current context is clearly not direct in the sense of being immediate: Like all visual routines, it is a dynamic temporally extended operation, realized via path tracing.\footnotemark

\footnotetext{Another contrast between the current experiments and classical Gibsonian proposals involves the degree to which affordances must involve oneself. For Gibson, affordances are central because of how they allow us to act, and thus affordances may differ for different people: A certain stairway with a large riser height may be perceived as \textit{climbable} (or a certain hefty stone may be perceived as \textit{graspable}) for an adult, but not for a young child. Here, in contrast, the navigational affordance was only between two “thirdperson” probes and did not directly involve oneself (though of course one may readily imagine oneself in such a maze, moving from 1 point to the other).}

Putting this same point in more operationalized terms, note that previous experiments involving affordances could not speak to this possibility because they nearly always involved affordances that seemed immediately perceptible --- for example, the properties of being “fall-off-able” or “bump-into-able” (\cite{gibson_ecological_1979}, p.~128), the climbability of stairs \parencite{warren_perceiving_1984}, or the passability of an aperture \parencite{warren_visual_1987}. In contrast, in the current experiments, the “navigability” of being able to trace from one probe to the other (without being blocked by an obstacle) cannot be immediately recovered without first executing the temporally extended routine of path tracing.

%%%%%%%%%%%%%%%%%%%%%%%%%%%%%%%%%%%%%%%%%%%%%%%%%%%%%%%%%%%%%%%%%%%%%%%%%%%
\subsection{Constraints on Generality}
%%%%%%%%%%%%%%%%%%%%%%%%%%%%%%%%%%%%%%%%%%%%%%%%%%%%%%%%%%%%%%%%%%%%%%%%%%%
The effects reported here generalized across multiple types of obstructions in otherwise-navigable paths between two probes --- including both walls of mazes (in familiar cultural artifacts) and arrays of discs (in unfamiliar scenes). However, there remain several open questions about the degree to which these results might generalize in other ways. Of course future work could explore such effects in different subject populations. Based on the possibility that visual routines and affordances both reflect core aspects of scene perception, however, we would predict relatively few qualitative and systematic differences across people --- and at least in the current data, we did not observe any salient systematic differences across age or gender. It might also be of interest for future work to explore how and whether such effects may depend on observers paying close and careful attention to the displays. (Of course, we cannot be sure about this without testing it directly, but our hunch is that this study is somewhere in the middle when it comes to the degree of attentiveness that is necessary. On one hand, subtle response–time measures often suffer from inattention, but on the other hand, the current results were robust enough to be observed in online samples --- who are often paying less focused attention than is possible to ensure during in-lab experiments.) Finally, it may also be especially important for future work to explore how these effects may or may not generalize to other types of affordances (beyond spatial affordances such as navigability) and visual routines (beyond path tracing).

%%%%%%%%%%%%%%%%%%%%%%%%%%%%%%%%%%%%%%%%%%%%%%%%%%%%%%%%%%%%%%%%%%%%%%%%%%%
\subsection{Conclusions}
%%%%%%%%%%%%%%%%%%%%%%%%%%%%%%%%%%%%%%%%%%%%%%%%%%%%%%%%%%%%%%%%%%%%%%%%%%%
It is perhaps unsurprising that the current sorts of experiments have not been previously conducted because key parts of the two foundational motivations for this project seem incompatible on their surfaces, in two closely related ways: (a) visual routines are inherently dynamic and temporally extended, while affordances are thought to be directly and immediately extracted, and (b) many affordances are extracted as a part of spontaneous perception itself, while visual routines are thought to be triggered only by specific goals. The current experiments suggest that divorcing these theoretical properties from each other may yield empirical benefits: Some dynamic visual routines may be spontaneously triggered by navigational affordances --- as in the visual perception of mazes.