%%%%%%%%%%%%%%%%%%%%%%%%%%%%%%%%%%%%%%%%%%%%%%%%%%%%%%%%%%%%%%%%%%%%%%%%%%%
\fancyChapter{Seeing from the Ground Up}[Spontaneous perception of ``causal history'' \\due to intuitive physics]
\label{chap:casual_hist}
%%%%%%%%%%%%%%%%%%%%%%%%%%%%%%%%%%%%%%%%%%%%%%%%%%%%%%%%%%%%%%%%%%%%%%%%%%%

We typically think of visual perception as providing us with representations of our present local environments.  But vision may also sometimes represent the causal \textit{past}, extracting how those environments got to be that way --- as when a shape with a jagged ``bite'' is represented as the full (unbitten) shape to which an event (biting) occurred.  Here we suggest that this perception of ``causal history'' is more prevalent than previously suspected, due to intuitive physics.  In a stack of blocks (or books, or dishes), for example, gravity entails that the bottom object was placed before higher objects.  Here we show that such ``historical'' relationships are spontaneously extracted during passive viewing, and influence perception in surprising ways.  Observers viewed a table on which a stack of two blocks appeared, (1) all at once, (2) with the bottom block appearing first, or (3) with the top appearing first --- and they simply reported on each trial whether the blocks appeared simultaneously or sequentially.  We reasoned that possibility (3) might be less naturally perceived, since it violates the causal history mandated by the underlying intuitive physics.  And indeed: when the bottom block appeared first, observers reliably perceived this sequential presentation; but when the top block appeared first, they were more likely to mistakenly perceive that blocks had appeared simultaneously --- as if the actual temporal offset and the gravity-inspired prior effectively cancelled out.  In addition, observers were more accurate for towers built ``from the ground-up'' than for actual simultaneous presentations (which were often misperceived as sequential, bottom-to-top) --- reflecting influences of causal history as well.  And these effects seemed specific to gravity-based intuitive physics, since they disappeared when the same stimuli appeared to be lying flat on the table.  These results collectively show how visual processing extracts causal history as a result of intuitive physics, and how such representations influence the perception of temporal order.

%%%%%%%%%%%%%%%%%%%%%%%%%%%%%%%%%%%%%%%%%%%%%%%%%%%%%%%%%%%%%%%%%%%%%%%%%%%
\section{Introduction}
%%%%%%%%%%%%%%%%%%%%%%%%%%%%%%%%%%%%%%%%%%%%%%%%%%%%%%%%%%%%%%%%%%%%%%%%%%%


What do we actually \textit{see} when we look at a given scene, such as the one in \Cref{fig:CausalHist_1}a?  Of course we perceive the current shape and its details --- in this case, a bendy novel object with jagged and smooth edges --- but does perception capture more than the here-and-now?  From the detail of the jagged edge, we may be able to extract the ``casual history'' of how this scene came to be --- a chunk of the shape seems to have been ``bitten'' off.  Such examples raise the possibility that the perception of shape can also entail the perception of time.  Perhaps, what we see isn’t just the present state of scenes, but also its \textit{past}.  

Previous work has demonstrated that people are indeed able to accurately assess, categorize, and group objects that share the same ``type'' of causal history.   For example, when presented with an array of novel 3D objects (that had each been subjected to some form of shape transformation such as getting ``twisted'', ``bent'', ``inflated/bloated'', ``hammered'', etc.), participants are able to accurately select objects that were subject to the same causal history transformation type \parencite{fleming_getting_2019, schmidt_visual_2019}.  This ability translates into real world objects as well, as participants accurately name and place stimuli into various causal history transformation categories, regardless of these transformations being applied to different materials (e.g.~wax, aluminum foil, cardboard; \cite{schmidt_identifying_2018}).  Additionally, in a study demonstrating that causal history influences judgments of shape representation \parencite{sprote_visual_2016}, participants reported symmetry axes of novel, irregular shapes with causal history (bitten shapes) to be overall more similar to those of corresponding shapes pre-causal transformation (whole shapes), and less similar to those of corresponding shapes that lacked causal history but still shared the concavities of the bitten shapes (smoothed shapes).  Even children (by age 5) seem to be able to judge the placement order of vertically stacked objects when explicitly asked “What came first?” or “What came last?” (e.g.~in a stack of hats, or a tower of ice cream; \cite{goulding_time_2025}).

\begin{figure}
    \centering
    \includegraphics[width=\textwidth]{figures/Causal2025/bitten_novel_shape_jagged.pdf}
    \caption
    {Two versions of a novel irregular shape, where (a) appears to have been bitten, while (b) appears to be its original (not causally transformed) form.}
    \label{fig:CausalHist_1}
\end{figure}

Leyton (\citeyear{leyton_inferring_1989}) was fascinated with the possibility that shape “can be used as a window into the past” (p.~1), and that “it is this history that is seen in them” (p.~2, emphasis added).  And while the aforementioned work clearly demonstrates the human ability to judge, categorize, and describe the causal history of an object, the implications for Leyton’s theories are limited by this very methodology --- participants were always explicitly asked about causal history.  In short, although we know that people are able to think about causal history as a part of higher-level cognition, are people also able to see causal history as a part of lower-level visual processing? 

Exceedingly little empirical work can speak to perception of causal history.  In the first (and arguably only) work demonstrating the perception of causal history \parencite{chen_perception_2016}, observers viewed very brief animations of squares undergoing two different transformations, either “intrusions” (as when a square-shaped clay is indented with a finger) or “impositions” (as when a piece of a square-shaped clay is cut out by a cookie-cutter).  Even though the animation was only two frames long (frame 1: whole square, frame 2: either intruded or imposed square), observers experienced illusory apparent motion more often when it was an intruded square than when it was an imposed square.  This effect was explained by way of causal history, as imposition transformations can only happen suddenly (as the cutout happens all-at-once regardless of location), but intrusions can happen gradually (as the clay is indented deeper and deeper until it reaches its final point).  In the only other empirical paper using a perception-based task (visual search, where participants searched for bitten cookie targets amongst whole cookie and misshapen cookie distractors), results on the possible influence of causal history were mixed \parencite{brenner_searching_2020}.  And in work on attention and causal history, it was observed that when viewing irregular novel shapes that are bitten, attention is allocated towards the remaining non-transformed regions, such that probes are detected faster (meanwhile corresponding complete, unbitten shapes do not results in the same effects; \cite{chen_causal_2021}).

Not only has the amount of empirical work on possible lower-level visual processing of causal history been limited, but the scope has been relatively limited as well: discussions of causal history primarily focus on affected singular objects and manipulations that distort the shape itself.  But the essence of causal history is to see “time” --- specifically, “the past” --- in a scene.  And surely, the rules of physics would be very informative to that past.  We thus propose that causal history is in fact far more prominent than previously described, as it is supported by our visual processing of intuitive physics.

%%%%%%%%%%%%%%%%%%%%%%%%%%%%%%%%%%%%%%%%%%%%%%%%%%%%%%%%%%%%%%%%%%%%%%%%%%%
\section{Intuitive Physics and Time}
%%%%%%%%%%%%%%%%%%%%%%%%%%%%%%%%%%%%%%%%%%%%%%%%%%%%%%%%%%%%%%%%%%%%%%%%%%%
In contrast to the research on causal history, research on \textit{intuitive physics} (how physics is processed in the mind) has been primarily explored through a different temporal lens --- focusing on recovering \textit{present} states or predicting \textit{future} ones.  For example, classic intuitive physics work asks participants to determine properties of some given scene, such as relative masses of two colliding objects (e.g.~\cite{todd_visual_1982, gilden_understanding_1989}).  Or in the context of “prediction”, wherein results are contingent on simulating forward in time, participants have been asked questions such as whether a block tower will fall (e.g.~\cite{hamrick_inferring_2016,battaglia_simulation_2013}), where liquid will flow (e.g.~\cite{bates_humans_2015}), or the trajectory of falling balls (e.g.~\cite{mccloskey_etal_1983}; see \cite{kubricht_intuitive_2017}, for a review).  Even for the few rare cases where participants are asked to recount the past of a scene by using intuitive physics, these studies have always relied on asking participants to \textit{explicitly} reason through what had occurred \parencite{beller_looking_2022}.

Similar to the progression of literature on causal history, it was only recently that  intuitive physics has expanded beyond the domain of higher-level reasoning and judgments, and into perception and attention.  In particular, even when intuitive physics information is irrelevant to the task at hand (or even detrimental to performance), observers take into account physical forces such as gravity (e.g.~\cite{wong_seeing_2023, mitko_visual_2023}), mass (e.g.~\cite{deeb_velocity_2024}), friction \parencite{nguyen_rotating_2024}, and stability (e.g.~\cite{wong_unconscious_2024, firestone_seeing_2017}).  For example, when shown images of objects covered by cloth, observers spontaneously account for the physical interactions between the soft material, the rigid object beneath the cloth, and gravity --- resulting in better change detection for changes to the structure beneath the cloth vs. those involving only the superficial folds of cloths, even when the latter were more extreme along several objective dimensions \parencite{wong_seeing_2023}.  In terms of attention, when viewing displays of rotating wheels, participants respond faster to probes that appear in locations where the wheel is \textit{about to} roll towards, with attended location shifting depending on fricative contact with surfaces \parencite{nguyen_rotating_2024}.  Similarly, when tasked with detecting color changes of a ball on a ramp, participants respond faster when the color change is simultaneously accompanied by the ball being displaced slightly further down the ramp --- a result that aligns with using intuitive physics to simulate the future location of ball \parencite{mitko_visual_2023}.  In this way, attention and perception of intuitive physics research has explored how objects and forces may act upon each other, primarily describing the \textit{current} and \textit{future} states of said scenes.  

%%%%%%%%%%%%%%%%%%%%%%%%%%%%%%%%%%%%%%%%%%%%%%%%%%%%%%%%%%%%%%%%%%%%%%%%%%%
\section{The Current Study: Causal History Meets Intuitive Physics}
%%%%%%%%%%%%%%%%%%%%%%%%%%%%%%%%%%%%%%%%%%%%%%%%%%%%%%%%%%%%%%%%%%%%%%%%%%%

Here, for the first time to our knowledge, we explore how perception extracts and represents scenes in terms of their causal history, based on intuitive physics --- demonstrating that causal history is far more prevalent than previously assumed or ever empirically explored.  Take a moment to view the scene depicted in \Cref{fig:CausalHist_2}a.  There are two blocks stacked on a table.  This type of stimulus, in the intuitive physics world, would usually entail a question about its (in)stability.  Here, we ask a different type of question, about its past: which block was placed first?  The answer is obvious.  No matter what, that stack of blocks (or even a hypothetical pile of books, or dishes) are subject to the rules of gravity --- thus, the causal history of such scenes \textit{must} consist of the bottom object being placed first, before the higher objects.  Now take a moment to view the scene depicted in \Cref{fig:CausalHist_2}b, and ask yourself the same question: which block was placed first?  This no longer has such a certain answer, as both blocks are laying flat on the table --- either the bottom or the top block could be placed first without violating the rules of physics.  

\begin{figure}
    \centering
    \includegraphics[width=\textwidth]{figures/Causal2025/sidebyside.pdf}
    \caption
    {Two images, both with a purple directly above a green block, but with vastly different implications for intuitive physics-based causal history.  While (a) must have been built from bottom-to-top, (b) does not have restrictions on build order.}
    \label{fig:CausalHist_2}
\end{figure}

Observers watched animations consisting of two blocks appearing atop a table.  These animations either involved all the blocks appearing (a) simultaneously, (b) sequentially, from bottom-to-top, or (c) sequentially, from top-to bottom.  Observers were given (what should have been) a straightforward task:  simply report whether the blocks had appeared \textit{simultaneously or sequentially}.  The sequential direction that the blocks appeared was therefore completely task-irrelevant, and similarly, any reasoning about intuitive physics or casual history was also unnecessary.  Yet despite this task-irrelevance, will we nevertheless observe that (1) causal history influences perception of temporal order for both simultaneous animations and sequential animations, and that (2) the underlying intuitive physics informs the nature of this casual past?  In essence, we have two competing factors in these experiments:  sensory information (the actual animation that unfolds) and the underlying intuitive physics-based causal history (mandating bottom-to-top).  Could this causal history be prominent enough to sometimes override sensory input?

In Expt.~1a, animations consisted of two blocks stacked one on top of the other (in a “From Front” view, as in \Cref{fig:CausalHist_2}a), such that the intuitive physics would necessitate a causal history of being built with the bottom block first.  In contrast, Expt.~1b consists of animations with the two blocks flat on the table (in a “From Above” view, as in \Cref{fig:CausalHist_2}b), such that the intuitive physics no longer necessitates a specific build order.  We then generalize these effects in several ways (Expts.~2-3), including with blocks that lean on each other (instead of being vertically stacked as in Expt.~1). Finally, Expt.~4 explores the nature of perceived temporal order of the “simultaneous” displays, showing that when participants mistakenly see simultaneous displays as sequential, they do indeed see it as “sequential bottom-up” --- as would be predicted by the underlying theory of physics-based causal history. Specifically, despite both intuitive physics and intuitive-based causal history being completely irrelevant to the task, would observers' perception of temporal order nevertheless be influenced? 

%%%%%%%%%%%%%%%%%%%%%%%%%%%%%%%%%%%%%%%%%%%%%%%%%%%%%%%%%%%%%%%%%%%%%%%%%%%
\section{Experiment 1a: Two Blocks Viewed from Front}
%%%%%%%%%%%%%%%%%%%%%%%%%%%%%%%%%%%%%%%%%%%%%%%%%%%%%%%%%%%%%%%%%%%%%%%%%%%

We first explored the possibility of physics-based causal history influencing temporal order perception using animations of two blocks stacked on each other, on top of a table, as depicted in \Cref{fig:CausalHist_2}a.  These animations either involved the two blocks appearing (a) simultaneously, (b) sequentially, from bottom-to-top, or (c) sequentially, from top-to-bottom.  Observers were tasked with simply reporting whether the blocks had appeared simultaneously or sequentially.

%%%%%%%%%%%%%%%%%%%%%%%%%%%%%%%%%%%%%%%%%%%%%%%%%%%%%%%%%%%%%%%%%%%%%%%%%%%
\subsection{Method}
%%%%%%%%%%%%%%%%%%%%%%%%%%%%%%%%%%%%%%%%%%%%%%%%%%%%%%%%%%%%%%%%%%%%%%%%%%%
%%%%%%%%%%%%%%%%%%%%%%%%%%%%%%%%%%%%%%%%%%%%%%%%%%%%%%%%%%%%%%%%%%%%%%%%%%%
\subsubsection{Participants}
%%%%%%%%%%%%%%%%%%%%%%%%%%%%%%%%%%%%%%%%%%%%%%%%%%%%%%%%%%%%%%%%%%%%%%%%%%%
100 observers ($M_\text{age} = 35.00$; 44 female, 56 male) participated using the Prolific online platform \parencite{palan_prolificacsubject_2018} for monetary compensation, with this preregistered sample size determined before data collection began.  Observers were excluded (with replacement) according to three preregistered criteria.  First, in a post-experimental debriefing phase, observers self-reported how well they paid attention (on a continuous scale, between $1 = \text{very distracted}$ and $100 = \text{very focused}$), and we excluded observers who self-reported an attention level at or below 80 ($n = 15$).  Second, we also excluded observers (who were not already excluded via criterion 1) whose overall accuracy was less than \qty{50}{\percent} ($n = 4$), or who only pressed the same button to respond throughout the entire experiment ($n = 0$).
%%%%%%%%%%%%%%%%%%%%%%%%%%%%%%%%%%%%%%%%%%%%%%%%%%%%%%%%%%%%%%%%%%%%%%%%%%%
\subsubsection{Apparatus}
%%%%%%%%%%%%%%%%%%%%%%%%%%%%%%%%%%%%%%%%%%%%%%%%%%%%%%%%%%%%%%%%%%%%%%%%%%%
After agreeing to participate, observers were redirected to a website where stimulus presentation and data collection were controlled via custom software written using a combination of HTML, CSS, JavaScript, PHP, and the jsPsych libraries \parencite{de_leeuw_jspsych_2023}.  Observers completed the experiment on either a laptop or desktop computer, with a minimum of a \qty{60}{\hertz} monitor. (Since the experiment was rendered on observers’ own web browsers, viewing distance, screen size, and display resolutions could vary dramatically, so we report stimulus dimensions below using pixel [\unit{\pixel}] values.)

%%%%%%%%%%%%%%%%%%%%%%%%%%%%%%%%%%%%%%%%%%%%%%%%%%%%%%%%%%%%%%%%%%%%%%%%%%%
\subsubsection{Stimuli}
%%%%%%%%%%%%%%%%%%%%%%%%%%%%%%%%%%%%%%%%%%%%%%%%%%%%%%%%%%%%%%%%%%%%%%%%%%%
All text, across the instructions and prompts, was presented in a modified version of jsPsych’s default CSS style: black text on a white background drawn in the “Open Sans” font, presented at a font size that is scaled to \qty{2.4}{\percent} of the participant’s viewport height.  Stimuli were constructed in Blender along with custom Python code.  7 unique stacks of blocks (consisting of two blocks each) were generated from a random selection of 6 colors (hex No.~36a4bd~\legendbox{36a4bd}, b9b932~\legendbox{b9b932}, ba9aad~\legendbox{ba9aad}, 30b630~\legendbox{30b630}, a997b7~\legendbox{a997b7}, ba983d~\legendbox{ba983d}), placed in one of two random orientations (horizontal or vertical), and random lengths (within the range of 83-\qty{119}{\pixel}, averaging \qty{106.28}{\pixel}; the shortest edge was always \qty{40}{\pixel}). To create these “From Front” view images, the camera was angled at \qty{81}{\degree} (where \qty{0}{\degree} is straight down, and \qty{180}{\degree} is straight up).  Blocks were placed onto a $600 \times \qty{338}{\pixel}$ background image of a white table (hex No.~b9b9b9~\legendbox{b9b9b9}) on a brown background (a light brown floor hex No.~4d301f~\legendbox{4d301f}, and a dark brown wall hex No.~261105~\legendbox{261105}).

%%%%%%%%%%%%%%%%%%%%%%%%%%%%%%%%%%%%%%%%%%%%%%%%%%%%%%%%%%%%%%%%%%%%%%%%%%%
\subsubsection{Procedure and Design}
%%%%%%%%%%%%%%%%%%%%%%%%%%%%%%%%%%%%%%%%%%%%%%%%%%%%%%%%%%%%%%%%%%%%%%%%%%%
Each trial began with an image of a table on a brown background appearing at the center of the display. After \qty{83.34}{\milli\second}, a stack of two blocks appeared on the table.  On one third of the trials, these two blocks appeared simultaneously, where the two blocks remained on screen for an additional \qty{33.33}{\milli\second}. On another one third of trials, the two blocks appeared sequentially, with the \textit{bottom} block appearing first, and then the top block appearing \qty{33.33}{\milli\second} later.  And on the remaining third of trials, the two blocks appeared sequentially, but this time with the \textit{top} appearing first, and then the bottom block appearing two frames \qty{33.33}{\milli\second} later.  This entire scene of blocks on the table then remained on screen for an additional \qty{166.67}{\milli\second} before disappearing.  Observers had been instructed ahead of time that after the animation finishes, they were to indicate (via key press) whether the scene had blocks that appeared sequentially one-by-one, or blocks that had appeared simultaneously all-at-once.  The trial would not conclude until a valid key press was provided.

Each observer completed a total of 42 trials: 3 display conditions (simultaneous vs. sequential bottom-to-top vs. sequential top-to-bottom) $\times\;7$ unique block stack stimuli $\times\;2$ orientations (normal vs. horizontally flipped).
%%%%%%%%%%%%%%%%%%%%%%%%%%%%%%%%%%%%%%%%%%%%%%%%%%%%%%%%%%%%%%%%%%%%%%%%%%%
\subsection{Results and Discussion}
%%%%%%%%%%%%%%%%%%%%%%%%%%%%%%%%%%%%%%%%%%%%%%%%%%%%%%%%%%%%%%%%%%%%%%%%%%%
As depicted in \Cref{fig:CausalHist_3}a, accuracy was significantly \textit{higher} on sequential bottom-to-top trials than on sequential top-to-bottom trials, as confirmed with a two-tailed paired t-test, \qty{84.64}{\percent} vs. \qty{49.57}{\percent}, $t(99) = 15.14$, $p < 0.0001$, $d = 1.51$.  Accuracy was also significantly \textit{higher} on sequential bottom-to-top trials than on simultaneous, as confirmed with a two-tailed paired t-test, \qty{84.64}{\percent} vs. \qty{71.29}{\percent}, $t(99) = 6.85$, $p < 0.0001$, $d = 0.69$.

\begin{figure}
    \centering
    \includegraphics[width=\textwidth]{figures/Causal2025/expt1_results_lines_cut.pdf}
    \caption
    [Average accuracy on each animation type (i.e. correctly identifying when the blocks appeared “all-at-once” as simultaneous, and correctly identifying when the blocks appeared “bottom-up” or “top-down” as sequential).]
    {Average accuracy on each animation type (i.e. correctly identifying when the blocks appeared “all-at-once” as simultaneous, and correctly identifying when the blocks appeared “bottom-up” or “top-down” as sequential.  (a) Performance on Expt.~1a, “From Front” images, where blocks are sitting upright on a table.  (b) Performance on Expt.~1b, “From Above” images, where blocks are both laid flat on the table.  Error bars reflect \qty{95}{\percent} confidence intervals after subtracting shared variance. ($^{**}p < .01$, $^{***}p < .001$)}
    \label{fig:CausalHist_3}
\end{figure}

%%%%%%%%%%%%%%%%%%%%%%%%%%%%%%%%%%%%%%%%%%%%%%%%%%%%%%%%%%%%%%%%%%%%%%%%%%%
\section{Experiment 1b: Two Blocks Viewed from Above}
%%%%%%%%%%%%%%%%%%%%%%%%%%%%%%%%%%%%%%%%%%%%%%%%%%%%%%%%%%%%%%%%%%%%%%%%%%%
If it is true that these differences in temporal order perception are due to the intuitive physics underlying the scene, then we would expect these differences in performance across conditions to be attenuated once the scene no longer necessitates being built from the bottom-up.  Thus, we conduct the same study as Expt.~1a, but now with all the blocks placed \textit{flat} on the table, as in \Cref{fig:CausalHist_2}b (in a “From Above” view, instead of appearing to be sitting upright \textit{on} the table, as in Expt.~1a’s “From Front” view).

%%%%%%%%%%%%%%%%%%%%%%%%%%%%%%%%%%%%%%%%%%%%%%%%%%%%%%%%%%%%%%%%%%%%%%%%%%%
\subsection{Method}
%%%%%%%%%%%%%%%%%%%%%%%%%%%%%%%%%%%%%%%%%%%%%%%%%%%%%%%%%%%%%%%%%%%%%%%%%%%
This experiment was identical to Expt.~1a except as noted.  100 new observers ($M_\text{age} = \qty{35.8}{years}$; 49 female, 48 male, 1 other, 2 preferred not to say) participated (with this sample size chosen to exactly match that of Expt.~1).  For these “From Above” view images, all block combinations (relative positions, colors, etc) were identical in Blender, but now the camera is angled at \qty{101}{\degree}.  Because of this new camera angle, the longest edge of each block now falls within the range of 83-\qty{121}{\pixel}, averaging \qty{99.36}{\pixel} (the shortest edge was always \qty{40}{\pixel}).  The background image here consists of a table with a white surface (hex b9b9b9~\legendbox{b9b9b9}), and a brown floorboard background.  All blocks, tables, and flooring in the “From Above” view have realistic shadows consistent with light coming from above.  Observers were excluded (with replacement) according to the same three preregistered criteria as Expt.~1a: self-reported attention level ($n=11$), overall accuracy being lower than \qty{50}{\percent} ($n=8$), and repetitive button presses ($n = 0$).

%%%%%%%%%%%%%%%%%%%%%%%%%%%%%%%%%%%%%%%%%%%%%%%%%%%%%%%%%%%%%%%%%%%%%%%%%%%
\subsection{Results and Discussion}
%%%%%%%%%%%%%%%%%%%%%%%%%%%%%%%%%%%%%%%%%%%%%%%%%%%%%%%%%%%%%%%%%%%%%%%%%%%
As depicted in \Cref{fig:CausalHist_3}b, the direction of effects were completely reversed compared to 1a: accuracy was significantly \textit{lower} on sequential bottom-to-top trials than on sequential top-to-bottom trials, as confirmed with a two-tailed paired t-test, \qty{68.57}{\percent} vs \qty{81.43}{\percent}, $t(99) = 7.92$, $p < 0.0001$, $d = 0.79$.  Accuracy was also significantly \textit{lower} on sequential bottom-to-top trials than on simultaneous, as confirmed with a two-tailed paired t-test, \qty{68.57}{\percent} vs. \qty{77.29}{\percent}, $t(99) = 3.25$, $p = 0.0016$, $d = 0.33$.

Additionally, we conducted the preregistered interaction effect analyses between Expt.~1a and Expt.~1b, comparing difference-of-differences in unpaired t-tests for both (a) sequential bottom-to-top vs. sequential top-to-bottom and (b) sequential bottom-to-top vs. simultaneous.  As depicted in \Cref{fig:CausalHist_3}, there was a significant interaction effect between Expt.~1a and Expt.~1b for (a) $t(198) = 6.66$, $p < 0.0001$, $d = .94$, and (b)~$t(198) = 16.94$, $p < 0.0001$, $d = 2.40$.

%%%%%%%%%%%%%%%%%%%%%%%%%%%%%%%%%%%%%%%%%%%%%%%%%%%%%%%%%%%%%%%%%%%%%%%%%%%
\section{Experiment 2a:  Two Blocks Viewed from Front (Background Persists)}
%%%%%%%%%%%%%%%%%%%%%%%%%%%%%%%%%%%%%%%%%%%%%%%%%%%%%%%%%%%%%%%%%%%%%%%%%%%
Results from the last two experiments substantiate the claim that the causal history of construction influences how observers’ perception of temporal order.  However, since the background image flashes on screen at the start of each trial, the different positioning of the background table across Expts.~1a and 1b (occupying the lower 3/4 versus the central 3/4 of the image, respectively) may have caused differences in attentional capture.  To ensure that the effects observed are indeed due to causal history, rather than attentional capture from the distracting flashing background, we conducted a conceptual replication where the background image remained on screen throughout the entire experiment, persisting across trials.\footnotemark

\footnotetext{In order to maintain consistency with Expts.~1a and 1b, where there were always a minimum of two distinct frames in the animation (first the background table, then blocks), we replaced the appearance of background table --- as it now persists across trials --- with the appearance of two brown poles, spaced out on either side of the block structure, which results in the new animations maintaining a minimum of two distinct frames (first the brown poles, then the blocks). These brown poles have no functional support relationship with the blocks themselves.}
%%%%%%%%%%%%%%%%%%%%%%%%%%%%%%%%%%%%%%%%%%%%%%%%%%%%%%%%%%%%%%%%%%%%%%%%%%%
\subsection{Method}
%%%%%%%%%%%%%%%%%%%%%%%%%%%%%%%%%%%%%%%%%%%%%%%%%%%%%%%%%%%%%%%%%%%%%%%%%%%
This experiment was identical to Expt.~1a except as noted.  100 new observers ($M_\text{age}=\qty{31.82}{years}$; 45 female, 53 male, 2 preferred not to say) participated (with this preregistered sample size chosen to exactly match that of Expt.~1).  These images now also included two brown (hex No. 725849~\legendbox{725849}) sticks that were $24 \times \qty{120}{\pixel}$, and the white table now spans $828 \times \qty{218}{\pixel}$ (correspondingly, the brown background is $1000 \times \qty{338}{\pixel}$).

\begin{figure}[ht]
    \centering
    \includegraphics[width=\textwidth]{figures/Causal2025/expt2_results_lines_cut.pdf}
    \caption
    [Average accuracy on each animation type for Expt.~2, where the background and table persist across trials, and extra distractor blocks are present.]
    {Average accuracy on each animation type for Expt.~2, where the background and table persist across trials, and extra distractor blocks are present.  (a) Performance on Expt.~2a, “From Front” images, where blocks are sitting upright on a table.  (b) Performance on Expt.~2b, “From Above” images, where blocks are both laid flat on the table.  Error bars reflect 95\% confidence intervals after subtracting shared variance. ($^{*}p < .05$,$^{***}p < .001$)}
    \label{fig:CausalHist_4}
\end{figure}

Each trial began with an image of a table on a brown background appearing at the center of the display.  Two thin sticks then appeared after \qty{800}{\milli\second}, at a randomly selected position while always keeping the outermost stick within \qty{174}{\pixel} of the center of the table (average displacement from center: $\qty{0.58}{\pixel}$).  After \qty{33.33}{\milli\second}, a stack of two blocks appeared on the table, in between the two thin sticks (always maintaining a minimum distance of \qty{23}{\pixel} between the closest stick and closest block edge).  All other timings are identical to Expt.~1a.  However, instead of the entire scene (table plus blocks) disappearing, the table and background would remain on screen, persisting into the next trial.

Observers were excluded (with replacement) according to three preregistered criteria. First, in a post-experimental debriefing phase, observers self-reported how well they paid attention (on a continuous scale, between $1 = \text{very distracted}$ and $100 = \text{very focused}$), and we excluded observers who self-reported an attention level at or below 80 ($n = 29$). Second, we also excluded observers (who were not already excluded via criterion 1) whose overall accuracy was less than \qty{60}{\percent} ($n = 71$), or who only pressed the same button to respond throughout the entire experiment ($n = 0$).



%%%%%%%%%%%%%%%%%%%%%%%%%%%%%%%%%%%%%%%%%%%%%%%%%%%%%%%%%%%%%%%%%%%%%%%%%%%
\subsection{Results and Discussion}
%%%%%%%%%%%%%%%%%%%%%%%%%%%%%%%%%%%%%%%%%%%%%%%%%%%%%%%%%%%%%%%%%%%%%%%%%%%

As depicted in \Cref{fig:CausalHist_4}a, accuracy was significantly \textit{higher} on sequential bottom-to-top trials than on sequential top-to-bottom trials, as confirmed with a two-tailed paired t-test, \qty{84.93}{\percent} vs. \qty{47.64}{\percent}, $t(99) = 15.81$, $p < 0.0001$, $d = 1.58$.  Accuracy was also significantly \textit{higher} on sequential bottom-to-top trials than on simultaneous, as confirmed with a two-tailed paired t-test, \qty{84.93}{\percent} vs. \qty{78.36}{\percent}, $t(99) = 3.41$, $p = 0.0009$, $d = 0.34$.\footnotemark

\footnotetext{Expts.~2a and 2b implemented a more stringent pre-registered exclusion criteria than Expts.~1a and 1b, resulting in a higher exclusion rate. We therefore conducted an additional analysis with exclusion criteria that matches Expt.~1a and 1b, resulting in a total of 146 non-excluded participants for Expt.~2a (with participants excluded for self-reported attention level [$n = 29$], overall accuracy being lower than 50\% [$n = 25$], and repetitive button presses [$n = 0$]), and $120$ non-excluded participants for 2b (with participants excluded for self-reported attention level [$n = 29$], overall accuracy being lower than \qty{50}{\percent} [$n = 10$], and repetitive button presses [$n = 0$]).  The pattern of results aligns with the data reported in the main text (with the preregistered stringent criteria).  Specifically, when evaluating interaction effects between Expt.~2a and 2b, we compared difference-of-differences in unpaired t-tests for both (a) sequential bottom-to-top vs. sequential top-to-bottom and (b) sequential bottom-to-top vs. simultaneous.  There was still a significant interaction effect between Expt.~2a and Expt.~2b with these less stringent exclusion criteria for (a) $t(264) = 15.91$, $p < 0.0001$, $d =1.96$, and (b) $t(264) = 2.99$, $p = 0.0030$, $d = 0.37$.}

%%%%%%%%%%%%%%%%%%%%%%%%%%%%%%%%%%%%%%%%%%%%%%%%%%%%%%%%%%%%%%%%%%%%%%%%%%%
\section{Experiment 2b:  Two Blocks Viewed from Above (Background Persists)}
%%%%%%%%%%%%%%%%%%%%%%%%%%%%%%%%%%%%%%%%%%%%%%%%%%%%%%%%%%%%%%%%%%%%%%%%%%%
Just like how we confirmed via Expt.~1b that the results of Expt.~1a were due to the intuitive physics of the scene, we applied the same manipulation here, with blocks from Expt.~2a placed flat on the table (in an “From Above” view, instead of “From Front” view as in Expts.~1a and 2a).
%%%%%%%%%%%%%%%%%%%%%%%%%%%%%%%%%%%%%%%%%%%%%%%%%%%%%%%%%%%%%%%%%%%%%%%%%%%
\subsection{Method}
%%%%%%%%%%%%%%%%%%%%%%%%%%%%%%%%%%%%%%%%%%%%%%%%%%%%%%%%%%%%%%%%%%%%%%%%%%%
This experiment was identical to Expt.~2a except as noted.  100 new observers ($M_\text{age}=\qty{33.79}{years}$; 49 female, 50 male, 1 other) participated (with this pre-registered sample size chosen to exactly match that of Expt.~2a).  For these “From Above” view images, all block combinations (relative positions, colors, etc) were identical in Blender, but now the camera is angled at \qty{101}{\degree}.  Because of this new camera angle, the longest edge of each block now falls within the range of 83-\qty{121}{\pixel}, averaging \qty{99.36}{\pixel} (the shortest edge was always \qty{40}{\pixel}), and each brown pole spans $20 \times \qty{118}{\pixel}$.  The minimum distance between the closest pole edge and the closest block edge was at least \qty{21}{\pixel}.  The background image here consists of a table with a $720 \times 243$ white surface (hex b9b9b9~\legendbox{b9b9b9}), and a brown floorboard background ($1000 \times \qty{338}{\pixel}$).  All blocks, tables, and flooring in the “From Above” view have realistic shadows consistent with light coming from above. 

Observers were excluded (with replacement) according to the same three preregistered criteria as Expt.~2a: self-reported attention level (n=29), overall accuracy being lower than \qty{60}{\percent} ($n=30$), and repetitive button presses ($n = 0$).

%%%%%%%%%%%%%%%%%%%%%%%%%%%%%%%%%%%%%%%%%%%%%%%%%%%%%%%%%%%%%%%%%%%%%%%%%%%
\subsection{Results and Discussion}
%%%%%%%%%%%%%%%%%%%%%%%%%%%%%%%%%%%%%%%%%%%%%%%%%%%%%%%%%%%%%%%%%%%%%%%%%%%
As depicted in \Cref{fig:CausalHist_4}b, the direction of effects were completely reversed compared to Expt.~2a: accuracy was significantly \textit{lower} on sequential bottom-to-top trials than on sequential top-to-bottom trials, as confirmed with a two-tailed paired t-test, \qty{72.86}{\percent} vs. \qty{82.05}{\percent}, $t(99) = 5.26$, $p < 0.0001$, $d = 0.53$.  Accuracy was also significantly \textit{lower} on sequential bottom-to-top trials than on simultaneous, as confirmed with a two-tailed paired t-test, \qty{72.86}{\percent} vs. \qty{78.50}{\percent}, $t(99) = 2.12$, $p < 0.05$, $d = 0.21$.

Additionally, we conducted the preregistered interaction effect analyses between Expt.~2a and Expt.~2b, comparing difference-of-differences in unpaired t-tests for both (a) sequential bottom-to-top vs. sequential top-to-bottom and (b) sequential bottom-to-top vs. simultaneous.  As depicted in \Cref{fig:CausalHist_4}, there was a significant interaction effect between Expt.~2a and Expt.~2b for (a) $t(198) = 15.71$, $p < 0.0001$, $d = 2.22$, and (b) $t(198) = 3.71$, $p = 0.0003$, $d = 0.52$.

%%%%%%%%%%%%%%%%%%%%%%%%%%%%%%%%%%%%%%%%%%%%%%%%%%%%%%%%%%%%%%%%%%%%%%%%%%%
\section{Experiment 3a: Leaning Blocks Viewed from Front}
%%%%%%%%%%%%%%%%%%%%%%%%%%%%%%%%%%%%%%%%%%%%%%%%%%%%%%%%%%%%%%%%%%%%%%%%%%%
Some recent work has discussed a Vertical-Attention Bias (VAB) such that attention is unevenly distributed towards the tops of objects \parencite{langley_vertical_2023, langley_children_2023}.  Our control conditions of the “From Above” view demonstrate that this VAB cannot explain our results.  However, to make sure our effect is due to how the physics act upon the scene (and not some other pattern in the specific relative placement of objects for our stimuli set, or some other other low-level factor), we created new animations consisting of blocks learning on each other, \textit{both} blocks now touching the table (as seen in \cref{fig:CausalHist_5}).  When a scene consists of one block (the “leaning” block) leaning against the other (the “supporting” block), the intuitive physics-based causal history would consist of the supporting block being placed \textit{first}, and the leaning block \textit{second}.
\begin{figure}
    \centering
    \includegraphics[width=\textwidth]{figures/Causal2025/sidebyside_leaning.pdf}
    \caption
    {Two images, both with a pink block leaned against a yellow block, but with vastly different implications for intuitive physics-based causal history.  While (a) must have been built with the yellow supporting block first, (b) does not have restrictions on build order.}
    \label{fig:CausalHist_5}
\end{figure}
%%%%%%%%%%%%%%%%%%%%%%%%%%%%%%%%%%%%%%%%%%%%%%%%%%%%%%%%%%%%%%%%%%%%%%%%%%%
\subsection{Method}
%%%%%%%%%%%%%%%%%%%%%%%%%%%%%%%%%%%%%%%%%%%%%%%%%%%%%%%%%%%%%%%%%%%%%%%%%%%
This experiment was identical to Expt.~1a except as noted.  100 new observers ($M_\text{age} = \qty{34.45}{years}$; 44 female, 55 male, 1 other) participated (with this sample size chosen to exactly match that of Expt.~1).  For each of the block pairs used in Expt.~1, they were reconfigured with one block randomly leaning (average tilt: \qty{63.83}{\degree}, ranging from \qty{30}{\degree}-\qty{119}{\degree}, where \qty{0}{\degree} is horizontal and \qty{90}{\degree} is vertical), such that the “leaning” block leans upon the other “supporting” block (which could either be horizontally or vertically oriented).  Accordingly, the different types of trials shall be referred to as: “sequential leaning-first” (where the leaning block appears first), “sequential supporting-first” (where the supporting block appears first), and “simultaneous” (where both blocks appear at the same time). Observers were excluded (with replacement) according to the same three preregistered criteria as Expt.~1a: self-reported attention level ($n = 27$), overall accuracy being lower than \qty{50}{\percent} ($n = 17$), and repetitive button presses ($n = 2$).
%%%%%%%%%%%%%%%%%%%%%%%%%%%%%%%%%%%%%%%%%%%%%%%%%%%%%%%%%%%%%%%%%%%%%%%%%%%
\subsection{Results and Discussion}
%%%%%%%%%%%%%%%%%%%%%%%%%%%%%%%%%%%%%%%%%%%%%%%%%%%%%%%%%%%%%%%%%%%%%%%%%%%
\begin{figure}[ht]
    \centering
    \includegraphics[width=\textwidth]{figures/Causal2025/expt3_results_lines_cut.pdf}
    \caption
    [Average accuracy on each animation type for Expt.~3, where one block is leaning on the other (instead of being vertically stacked as in Expt.~1 and 2).]
    {Average accuracy on each animation type for Expt.~3, where one block is leaning on the other (instead of being vertically stacked as in Expt.~1 and 2).  (a) Performance on Expt.~3a, “From Front” images, where blocks are sitting upright on a table.  (b) Performance on Expt.~3b, “From Above” images, where blocks are both laid flat on the table.  Error bars reflect \qty{95}{\percent} confidence intervals after subtracting shared variance. ($^{*}p < .05$; $^{**}p < .01$;  $^{***}p < .001$)}
    \label{fig:CausalHist_6}
\end{figure}
As depicted in \Cref{fig:CausalHist_6}a, accuracy was significantly \textit{higher} on sequential supporting-first trials than on sequential leaning-first trials, as confirmed with a two-tailed paired t-test, \qty{80.00}{\percent} vs. \qty{68.57}{\percent}, $t(99) = 6.98$, $p < 0.0001$, $d = 0.70$.  Accuracy was also significantly \textit{higher} on sequential supporting-first trials than on simultaneous trials, as confirmed with a two-tailed paired t-test, \qty{80.00}{\percent} vs. \qty{74.29}{\percent}, $t(99) = 2.57$, $p = 0.0115$, $d = 0.26$.

%%%%%%%%%%%%%%%%%%%%%%%%%%%%%%%%%%%%%%%%%%%%%%%%%%%%%%%%%%%%%%%%%%%%%%%%%%%
\section{Experiment 3b: Leaning Blocks Viewed From Above}
%%%%%%%%%%%%%%%%%%%%%%%%%%%%%%%%%%%%%%%%%%%%%%%%%%%%%%%%%%%%%%%%%%%%%%%%%%%
We conduct the same manipulation as Expt.~1b and 2b, with all the blocks placed \textit{flat} on the table (in a “From Above” view, instead of appearing to be sitting \textit{upright} on the table, as in the Expt.~1a and 2a’s “From Front” view).

%%%%%%%%%%%%%%%%%%%%%%%%%%%%%%%%%%%%%%%%%%%%%%%%%%%%%%%%%%%%%%%%%%%%%%%%%%%
\subsection{Method}
%%%%%%%%%%%%%%%%%%%%%%%%%%%%%%%%%%%%%%%%%%%%%%%%%%%%%%%%%%%%%%%%%%%%%%%%%%%
This experiment was identical to Expt.~3a except as noted.  100 new observers ($M_\text{age} = \qty{35.03}{years}$; 48 female, 51 male, 1 other) participated (with this sample size chosen to exactly match that of Expt.~3a).  For these “From Above” view images, all block combinations (relative positions, colors, etc) were identical in Blender, but now the camera is angled at \qty{101}{\degree}.  The table and brown background were identical to Expt.~1b.

%%%%%%%%%%%%%%%%%%%%%%%%%%%%%%%%%%%%%%%%%%%%%%%%%%%%%%%%%%%%%%%%%%%%%%%%%%%
\subsection{Results and Discussion}
%%%%%%%%%%%%%%%%%%%%%%%%%%%%%%%%%%%%%%%%%%%%%%%%%%%%%%%%%%%%%%%%%%%%%%%%%%%
As depicted in \Cref{fig:CausalHist_6}b, effects were modulated compared to Expt.~3a.  Accuracy was still significantly \textit{higher} on sequential supporting-first trials than on sequential leaning-first trials, as confirmed with a two-tailed paired t-test, \qty{79.42}{\percent} vs. \qty{74.43}{\percent}, $t(99) = 3.24$, $p = 0.0016$, $d = 0.32$.  There were no longer significant differences on sequential supporting-first trials than on simultaneous, as confirmed with a two-tailed paired t-test, \qty{79.42}{\percent} vs. \qty{81.71}{\percent}, $t(99) = 1.15$, $p = 0.2545$, $d = 0.11$.  Critically, as confirmed by preregistered interaction effect analyses between Expt.~3a and Expt.~3b, comparing difference-of-differences in unpaired t-tests for both (a) sequential supporting-first vs. sequential leaning-first and (b) sequential supporting-first vs. simultaneous, there was a significant interaction effect between Expt.~3a and Expt.~3b for (a) $t(198) = 2.86$, $p = 0.0047$, $d = 0.40$, and (b) $t(198) = 2.68$, $p = 0.0080$, $d = 0.38$.



%%%%%%%%%%%%%%%%%%%%%%%%%%%%%%%%%%%%%%%%%%%%%%%%%%%%%%%%%%%%%%%%%%%%%%%%%%%
\section{Experiment 4: Single Trial}
%%%%%%%%%%%%%%%%%%%%%%%%%%%%%%%%%%%%%%%%%%%%%%%%%%%%%%%%%%%%%%%%%%%%%%%%%%%
Across prior Expts.~1a and 2a, we consistently  observed that the performance on “simultaneous” trials was worse than “sequential bottom-to-top” trials.  We have been attributing this attenuation in performance to observers mistakenly perceiving the “simultaneous” trials as “sequential bottom-to-top”.  In this experiment, we directly test this assumption by conducting a single trial version of Expts.~1a and 1b, wherein if a participant mistakenly perceived a “simultaneous” trial as a “sequential” one, they were subsequently asked to specify the direction in which they saw the blocks appear as animation unfolded (i.e., whether it was bottom-to-top, or top-to-bottom).  If they misperceive the From Front “simultaneous” trials as bottom-to-top more often than top-to-bottom, but not so for From Above “simultaneous” trials, this would suggest that causal history directly impacts temporal perception. 

%%%%%%%%%%%%%%%%%%%%%%%%%%%%%%%%%%%%%%%%%%%%%%%%%%%%%%%%%%%%%%%%%%%%%%%%%%%
\subsection{Method}
%%%%%%%%%%%%%%%%%%%%%%%%%%%%%%%%%%%%%%%%%%%%%%%%%%%%%%%%%%%%%%%%%%%%%%%%%%%
This experiment was identical to Expts.~1a and 1b in combination except as noted. 110 new observers ($M_\text{age} = \qty{32.52}{years}$; 61 female, 49 male) participated.  Half of the observers were shown a single “simultaneous” trial of the “From Front” view (Expt.~1a), and the other half were shown a single “simultaneous” trial of the “From Above” view (Expt.~1b).  After the animation finished, they were first asked, “Did the items in the animation (besides the background) appear all-at-once or one-at-a-time?”, and were to click on one of two options to respond.  If the observer answered that they saw this animation as “sequential” (the incorrect answer), then they were prompted with a further question: “Did they appear from top-to-bottom, or bottom-to-top?”, and were to click on one of the two options to respond.  Observers were excluded (with replacement) if they did not mistakenly answer “sequential” ($n = 94$). 

%%%%%%%%%%%%%%%%%%%%%%%%%%%%%%%%%%%%%%%%%%%%%%%%%%%%%%%%%%%%%%%%%%%%%%%%%%%
\subsection{Results and Discussion}
%%%%%%%%%%%%%%%%%%%%%%%%%%%%%%%%%%%%%%%%%%%%%%%%%%%%%%%%%%%%%%%%%%%%%%%%%%%
On “From Front” simultaneous trials where observers misperceived the animation as sequential, 50/55 (\qty{90.91}{\percent}) answered that it was specifically “bottom-to-top”.  In contrast, on “From Above” simultaneous trials where observers misperceived the animation as sequential, only 34/55 (\qty{61.82}{\percent}) answered that it was specifically “bottom-to-top”.  This was a significant difference:  $\chi^2 (1, N = 110) = 12.89$, $p = 0.0003$, effect-size index $w = 0.34$.  In short, once the intuitive physics of the scene shift (such that it no longer necessitates being built from the ground up in “From Above” trials), observers are no longer biased to see the simultaneous animation as bottom to top --- as would be dictated by causal history.

%%%%%%%%%%%%%%%%%%%%%%%%%%%%%%%%%%%%%%%%%%%%%%%%%%%%%%%%%%%%%%%%%%%%%%%%%%%
\section{General Discussion}
%%%%%%%%%%%%%%%%%%%%%%%%%%%%%%%%%%%%%%%%%%%%%%%%%%%%%%%%%%%%%%%%%%%%%%%%%%%
The central theme of these four studies is straightforward: when viewing a scene, we spontaneously represent the causal history of how that scene came to be, affecting our temporal perception. Specifically, we see evidence that perception extracts and represents scenes in terms of their causal history --- critically based on the intuitive physics underlying the scene.  This work shows that even when seeing one of the simplest scenes possible (two blocks stacked on a table), we spontaneously represent the scene in terms of its causal history: that the blocks  must have been built from the ground-up due to rules of gravity and support.  We found the following patterns across Expts.~1a, 2a, and 3a:  (1) a boost in performance when the animation aligns with the intuitive physics-based causal history (i.e. blocks appearing from bottom-to-top, or with the supporting block first), (2) a deficit in performance when the animation conflicts with causal history (i.e. blocks appearing from top-to-bottom, or with the leaning block first), and (3) a deficit in performance on simultaneous trials, mistakenly being seen as sequential (confirmed by Expt.~4 to be based in causal history, as they are specifically misperceived as "bottom-to-top". Expts.~1b, 2b, and 3b further confirmed that these effects were contingent on the physical forces acting on the objects in the scene. Specifically, when the blocks were laid flat on the table (i.e., in the “From Above” view instead of the “From Front” view), all effects were attenuated since the external physical forces no longer played a role in how the block tower should be constructed.\footnote{Here we utilized the "From Above" viewpoint as a control, to change the way gravity acts upon the blocks. This offers advantages over other controls that change how gravity acts upon the blocks (i.e. a "inversion" control where where the tower and blocks are inverted) in two key ways: (a) the "From Above" view maintains all positions of blocks that were presented in the "From Front" view, both relative to each other and relative to the top and bottom of the screen, and (b) the "From Above" viewpoint is still a familiar type of scene to observers, whereas an inversion control would not only be physically be impossible, but also be a type of scene that observers have (likely) never encountered before. The "From Above" control thus effectively changes how gravity acts upon the blocks while maintaining relative block placement, and maintains a familiar scene.} 

The effects in Expts.~1 and 2 were especially robust in three ways.   First, they were highly statistically significant, with critical interaction p-values far from the traditional significance boundaries:  with critical interaction p-values $\leq 0.0003$ (all but one being $\leq 2\times10^{-10}$), and all critical main effect t-tests (Front View) at p < 0.0009 (all but one being $\leq 2\times10^{-10}$). Second, effect sizes for these interactions were large --- averaging $d = 1.52$, all greater than $d = 0.53$, and three key tests even exceeding $d = 0.94$.  This makes it notable, then, that while Expt.~3a still maintained a highly significant main effect for supporting-first vs. leaning-first ($p \leq 0.0001$, $d = 0.70$), the supporting-first vs. simultaneous effect was relatively weaker, but still highly significant (main effect $p = 0.01$, $d = 0.26$, interaction with Expt.~3b $p = 0.008$,  $d =  0.38$). We speculate that these Expt.~3a results reflect an additional dimension, beyond performance simply being higher on trials aligned with causal history. Intuitively, in a vertical stack, it's impossible to place a top block before placing the bottom block. However, for leaning blocks, it's theoretically possible for the leaning block to be placed first, begin to topple, but then have the supporting block quickly placed to prevent falling (as often happens in real life when objects begin to slip and are caught). This could explain why the "supporting block first" sequence maintains a statistically significant advantage, though not as pronounced as the "bottom-to-top" advantage in vertical stacking --- the order is not mandatory in the same way that it is for vertical block stacking.  

These results can thus collectively be interpreted in terms of the following: Upon viewing a scene, we spontaneously form representations of its underlying causal history (in this case, as supported by intuitive physics, to be “bottom-to-top” or “leaning-first”).  Look back now at \Cref{fig:CausalHist_2}a.  Seeing a scene not only involves forming visual representations of the colors and shapes, but also, critically, of blocks supporting each other --- and the necessary placement order of those blocks.  This underlying causal history representation explains why “top-to-bottom” trials consistently have the worst performance --- where top-to-bottom is mistakenly perceived as appearing simultaneously --- as if the actual animation’s temporal offset and the gravity-inspired causal history ‘prior’ effectively canceled each other out\footnotemark.  And as explored in Expt.~4, this causal history prior also influences “simultaneous” trials to be misperceived as their gravity-aligned order instead. 

\footnotetext{We also note that the main effect in "From Front" views (where performance on bottom-up trials was better than top-down trials), was qualitatively flipped in the control "From Above" views (where performance on top-down trials was better than bottom-up). We had not predicted this complete inversion of the effect, just that it would have been significantly modulated. This inversion could be due to a variety of co-occurring factors in the "From Above" view, such as (a) observers no longer getting the causal-history aligned performance boost to bottom-up trials, (b) observers no longer having causal history conflict with the top-down trials, and (c) the general tendency to allocate more attention to the tops of objects (e.g.~\cite{langley_vertical_2023}), which would further boost performance on top-down trials. Either way, this complete inversion demonstrates the effectiveness of changing how physics (gravity) is perceived to act upon the scene.}

This was the first demonstration of spontaneous intuitive physics-based causal history, as observers form representations of how the scene came to be despite it being completely task-irrelevant.  In fact, as observed in Expt.~1b, 2b, and 3b, overall performance would have been improved if casual history wasn’t taken into account. 

%%%%%%%%%%%%%%%%%%%%%%%%%%%%%%%%%%%%%%%%%%%%%%%%%%%%%%%%%%%%%%%%%%%%%%%%%%%
\subsection{On the Field of Causal History}
%%%%%%%%%%%%%%%%%%%%%%%%%%%%%%%%%%%%%%%%%%%%%%%%%%%%%%%%%%%%%%%%%%%%%%%%%%%
This study contributes to the field of causal history in two unique ways.  First, this study demonstrates that causal history is far more prevalent than previously explored, as it can now interface with the entire field of intuitive physics.  For example, in the (only) other demonstration of causal history in perception, Chen \& Scholl (\citeyear{chen_perception_2016}) utilized two-shot displays of a shape undergoing a transformation (“intrusion” and “imposition”).  Although familiar, those sorts of displays are relatively rare when compared to the ubiquity of the sorts of displays utilized here.  In a given room, there may be one or two instances of a ripped paper, a dented bottle, or a half-eaten sandwich.  In contrast, the very same room contains many instances of laptops and keyboards sitting on tables, books leaning against each other, food placed on plates, cups on coasters, etc.   Second, this study demonstrates that causal history is far more prevalent in a completely different way: representing history is not limited to distortions of a singular object (or shape), instead, it also includes how objects in a scene relate to each other.  This work may thus have implications for the underlying mechanisms of results in studies on object relations, such as those found in Hafri et al. (\citeyear{hafri_phone_2024}), wherein participants seem to spontaneously extract abstract relational properties (e.g.~“containment”) when viewing scenes such as a phone inside a basket.  Perhaps these effects could also be explained by way of causal history perception --- that the underlying intuitive physics necessitate placing the basket before the phone.

Future work can attempt to further generalize the ideas introduced here in several ways.  Here we examined one kind of intuitive physics: of support and gravity.  Similarly, there was only one type of stimulus (block towers).  Although various physics-based relations of these block towers were tested (e.g.~Expt.~3), there are many types of ways to observe intuitive physics in the world around us (such as rolling balls, broken walls, etc).  Further exploration would be warranted to understand possible reasons behind why causal history representations are spontaneously extracted from a scene.  We speculate that it would be evolutionarily beneficial to know what forces --- and whether or not those forces included \textit{agents} --- are present in a given environment, and that causal history enables quick extraction of these relevant features. 

This work is a prime demonstration of how we not only extract what is currently happening in a scene, but also what happened to get here.  Leyton (\citeyear{leyton_inferring_1989}, p.~1) wrote that “The shape of an object often seems to tell us something about the object’s history,” but here we broaden the scope beyond the shape or distortions of a singular object: the underlying intuitive physics and object relations tell us something about the scene’s history.  And importantly, this work demonstrates that we are not limited to reasoning about the past, but that we also can spontaneously extract and form these causal history representations.   Our visual system is inherently attuned to the past --- perceiving not just what \textit{is}, but how it possibly \textit{came to be}.
