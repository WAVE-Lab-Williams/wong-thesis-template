%%%%%%%%%%%%%%%%%%%%%%%%%%%%%%%%%%%%%%%
% Abstract Page
%%%%%%%%%%%%%%%%%%%%%%%%%%%%%%%%%%%%%%%
\newcommand{\abstractBody}{What is the purpose of visual perception?  The most common explanations  suggest that “seeing” is for answering the question ‘\textit{What’s out there?}’~--- giving us information about the features and objects in the current local environment.  This dissertation, in contrast, suggests a different approach: seeing is also for answering the question ‘\textit{What’s happening}?’~--- and the related questions ‘\textit{What just happened?}’~and ‘\textit{What’s about to happen?}’.  The critical difference is that while the more common answer seems implicitly \textit{static}, these newer answers are intrinsically \textit{dynamic}.   This dissertation proposes that visual processing represents even certain static images in terms of rich, dynamic representations.  I make this point via three empirical case studies, involving (1) intuitive physics, (2) causal history, and (3) navigational affordances.   The first example explores how \textit{intuitive physics} is taken into account even in visual processing: when viewing static images of objects covered by soft materials (e.g. cloths), observers spontaneously form dynamic representations of underlying physical interactions (between gravity, the cloth, and the rigid object beneath the cloth).   These representations then have powerful influences on visual attention and memory: observers are better at detecting changes to the deep underlying scene structure (the object beneath the cloth), compared to changes involving only the superficial folds of the cloth --- even when the latter were objectively more extreme along several dimensions.  A second example explores how perceiving collections of shapes also involves representing their \textit{causal history}.  When viewing static images of blocks stacked on top of each other, observers spontaneously form dynamic representations of the \textit{past}, which then dramatically influence current percepts in accord with intuitive physics (where such structures must be built ‘from the ground-up’): when blocks appear sequentially from top to bottom, observers mistakenly perceive them as appearing simultaneously --- and when blocks appear simultaneously, observers mistakenly perceive them as appearing from bottom to top.  In a final example, I explore novel connections between two prominent themes in our field: \textit{affordances} and \textit{visual routines}.  When viewing static images of maze-like stimuli (or scenes filled with obstacles), observers spontaneously engage in ‘mental path tracing’: when comparing two probes in such scenes, response times depend on the their ‘pathwise’ distance from each other, and not simply their Euclidean separation --- even when the paths are completely task-irrelevant.  Collectively, this work demonstrates that perception forms rich dynamic representations even of static scenes: we see what \textit{matters} --- visual representations of a scene’s deep underlying structure, its inferred past, and its likely future. }


%%%%%%%%%%%%%%%%%%%%%%%%%%%%%%%%%%%%%%%%%%%%%%%%%%%%%%%%%%%%%
% Page Layout
%%%%%%%%%%%%%%%%%%%%%%%%%%%%%%%%%%%%%%%%%%%%%%%%%%%%%%%%%%%%%
\begin{center}
    \rule{4in}{0.4pt}\\[0.2cm]
    {\Large \textit{Abstract}}\vspace{-0.2cm}\\
    \rule{4in}{0.4pt}\\[1.5cm]
    
    {\large \thesisTitle}\\[1cm]
    
    {\yourName}\\[1cm]
    
    {\yourYear}
\end{center}
\vspace{1.8cm}

\abstractBody

